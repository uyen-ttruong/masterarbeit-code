\subsection{Verteilung der Energieeffizienzklassen in Bayern}
Wie bereits in Abschnitt \ref{sec:EPC} angegeben, haben Energieausweise für Wohngebäude einen direkten Einfluss auf die Betriebskosten und den Wert von Immobilien. Es gibt derzeit keine offiziellen Statistiken von staatlichen Behörden über die Verteilung von Energieeffizienzklassen in Bayern. Laut \textcite{mcmakler2022}, einem führenden Maklerunternehmen in Deutschland, liegt Bayern jedoch an erster Stelle bei Wohnimmobilien mit positiven Energiekennwerten. In Bayern sind etwa 18\% der Wohnimmobilien energieeffizient (A+, A oder B), während 36\% schlechte Energiekennwerte haben \parencite{mcmakler2022}. Der Grund dafür liegt darin, dass die durchschnittliche Wohnimmobilie in Bayern erst 1991 gebaut wurde, was einen großen Unterschied im Vergleich zu anderen Bundesländern darstellt. Tabelle \ref{tab:epc_bayern} zeigt die Verteilung der Energieeffizienzklassen in Bayern.
\begin{table}[htbp]
    \centering
    \caption{Prozentuale Verteilung der Energieeffizienzklassen in Bayern}
    \label{tab:epc_bayern}
    \begin{tabularx}{\textwidth}{>{\raggedright\arraybackslash}X >{\centering\arraybackslash}X >{\centering\arraybackslash}X >{\centering\arraybackslash}X}
        \toprule
        & \textbf{A+, A, B} & \textbf{C, D, E} & \textbf{F, G, H} \\
        \midrule
        \textbf{Prozentanteil} & 18\% & 46\% & 36\% \\
        \bottomrule
    \end{tabularx}
\end{table}

Durch die Anwendung dieser Verteilung kann die \ac{EPC}-Verteilung in unser Hypothekenportfolio integriert werden, um eine möglichst realistische Darstellung des Portfolios zu erreichen.

