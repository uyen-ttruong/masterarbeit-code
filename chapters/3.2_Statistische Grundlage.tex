\subsection{Statistische Grundlage}
\subsubsection{Jährliche Überschreitungswahrscheinlichkeit}\label{sec:AEP}
Der Begriff des 100-jährlichen Ereignisses oder HQ\textsubscript{100}, wie in Abschnitt \ref{sec:HQ} erläutert, ist im öffentlichen Diskurs weit verbreitet. Diese Terminologie kann zu Missinterpretationen führen, da sie impliziert, dass solche Ereignisse nur alle 100 Jahre auftreten. Um dieses Missverständnis zu vermeiden, verwendet man in den Geowissenschaften das Konzept der \ac{AEP}, auch bekannt als Jährliche Überschreitungswahrscheinlichkeit \parencite{uswrc1981} .
Die \ac{AEP}, basierend auf der Theorie der unabhängigen Wahrscheinlichkeit, quantifiziert die jährliche Auftrittswahrscheinlichkeit von Extremereignissen als Prozentwert
Die Formel für die AEP lautet:
\begin{equation}
    \text{AEP} = 1 - \left(1 - \frac{1}{T}\right)^n
    \end{equation}
Wobei:
\begin{itemize}
\item T: Wiederkehrperiode in Jahren
\item n: Anzahl der betrachteten Jahre
\end{itemize}
Für ein einzelnes Jahr (n = 1) vereinfacht sich die Formel zu:
\begin{equation}
\text{AEP} = \frac{1}{T}
\end{equation}

Wie aus Tabelle \ref{tab:wiederkehrperiode} ersichtlich wird, hat ein Hochwasserereignis mit einer Wiederkehrperiode von 100 Jahren eine jährliche Auftrittswahrscheinlichkeit von 1\%. Ein Hochwasserereignis mit einer Wiederkehrperiode von 2 Jahren weist eine jährliche Auftrittswahrscheinlichkeit von 50\% auf.
\begin{table}[htbp]
    \centering
    \small  % Điều chỉnh kích thước font chữ cho bảng nếu cần
    \caption{Wiederkehrperiode und Jährliche Überschreitungswahrscheinlichkeit mit HQ\textsubscript{T\textsubscript{n}}}
    \label{tab:wiederkehrperiode}
    \begin{tabularx}{1.0\textwidth}{>{\centering\arraybackslash}X >{\centering\arraybackslash}X >{\centering\arraybackslash}X>{\centering\arraybackslash}X}
        \toprule  % Dòng kẻ đậm trên cùng
        \textbf{HQ\textsubscript{T\textsubscript{n}}} & \textbf{Wiederkehrperiode (Jahre)} & \textbf{Jährliche Überschreitungswahrscheinlichkeit} \\
        \midrule  % Dòng kẻ nhạt hơn dưới tên các cột
        HQ\textsubscript{2} & 2 & 50\% \\
        HQ\textsubscript{10} & 10 & 10\% \\
        HQ\textsubscript{25} & 25 & 4\% \\
        HQ\textsubscript{50} & 50 & 2\% \\
        HQ\textsubscript{100} & 100 & 1\% \\
        HQ\textsubscript{500} & 500 & 0,2\% \\
        HQ\textsubscript{1000} & 1000 & 0,1\% \\
        \bottomrule  % Dòng kẻ đậm dưới cùng
    \end{tabularx}
\end{table}
\FloatBarrier

\subsubsection{Stichprobenumfang }
Stichprobenverfahren werden genutzt, um die Verteilung der Merkmale in der Gesamtheit zu schätzen. Zur Gewährleistung der Repräsentativität bei begrenzter Teilmengengröße sind präzise statistische Kriterien erforderlich. Diese umfassen die Berechnung des Stichprobenumfangs und die Wahl der geeigneten Methode.

Für die Berechnung des erforderlichen Stichprobenumfangs bei einer theoretisch unendlichen Grundgesamtheit wird die Cochran-Formel (\cite{cochran1953sampling}) herangezogen. Die Cochran-Formel lautet:

\begin{equation}
n = \frac{Z^2 \cdot P(1 - P)}{\varepsilon^2}
\label{eq:cochran}
\end{equation}

Hierbei repräsentiert $n$ den initialen Stichprobenumfang. $Z$ steht für den Z-Wert des gewählten Konfidenzintervalls. $P$ bezeichnet die erwartete Wahrscheinlichkeit des untersuchten Merkmals, und $\epsilon$ repräsentiert den tolerierten Fehler.

Für Populationen mit bekannter, endlicher Größe wird die Cochran-Formel modifiziert. Die Originalformel \ref{eq:cochran} dient als Eingabeparameter für die folgende Gleichung \ref{eq:finite_population}. Diese berechnet den erforderlichen Stichprobenumfang für ein Portfolio der Größe $N$:

\begin{equation}
n' = \frac{n}{1 + \frac{n - 1}{N}}
\label{eq:finite_population}
\end{equation}

In dieser Gleichung steht $n'$ für die angepasste Stichprobengröße einer begrenzten Population. $N$ repräsentiert die Gesamtpopulationsgröße.