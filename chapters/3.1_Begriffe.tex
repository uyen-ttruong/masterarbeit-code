
\subsection{Wesentliche Begriffe}
\subsubsection{Beleihungsauslauf}
Der Beleihungsauslauf, auch als Loan-to-Value-Ratio bekannt, ist eine zentrale Kennzahl in der Immobilienfinanzierung \parencite{BelWertV_3}. Er wird definiert als:
\begin{equation}
    \text{Beleihungsauslauf} = \frac{\text{Darlehensbetrag}}{\text{Beleihungswert}}
    \label{eq:ltv}
\end{equation}

\noindent wobei:
\begin{itemize}
    \item Der Darlehensbetrag die Höhe des Kredits repräsentiert.
    \item Der Beleihungswert den langfristig realisierbaren Wert der Immobilie darstellt, unabhängig von kurzfristigen Marktschwankungen.
\end{itemize}

Ein niedrigerer Beleihungsauslauf impliziert ein geringeres Ausfallrisiko für den Kreditgeber. Konsequenterweise führt ein höherer Beleihungsauslauf in der Regel zu ungünstigeren Darlehenskonditionen für den Kreditnehmer.

Zur Veranschaulichung des Konzepts wird ein Beispiel einer Immobilienfinanzierung betrachtet:
Eine Immobilie wird von einem Sachverständigen mit einem Beleihungswert von 500.000 € eingeschätzt. Von der Person, die das Haus kaufen möchte, wird ein Darlehen in Höhe von 275.000 € beantragt.
Der Beleihungsauslauf wird gemäß Gleichung \ref{eq:ltv} folgendermaßen berechnet:
\begin{equation}
    \text{Beleihungsauslauf} = \frac{\text{Darlehensbetrag}}{\text{Beleihungswert}} = \frac{275.000 \mbox{\texteuro}}{500.000 \mbox{\texteuro}} = 0,55 = 55\%
\end{equation}
Es wird festgestellt, dass der Beleihungsauslauf in diesem Fall 55\% beträgt. Dies bedeutet, dass 55\% des von der Bank anerkannten Immobilienwertes durch das Darlehen finanziert werden. Die restlichen 45\% müssen durch Eigenkapital oder andere Mittel aufgebracht werden.
Ein Beleihungsauslauf von 55\% wird allgemein als vorsichtig eingestuft. Es kann angenommen werden, dass bei einem solchen Wert günstigere Darlehenskonditionen gewährt werden, da das Ausfallrisiko für die Bank als relativ gering betrachtet wird.
\subsubsection{Transitorische Risiken}
Transitorische Risiken sind Risiken, die temporär oder vorübergehend auftreten und in der Regel keine langfristigen Auswirkungen haben. Der Begriff wird in verschiedenen Kontexten verwendet, neben ESG und Klimawandel auch bei wirtschaftlichen, finanziellen und geschäftlichen Überlegungen. Hier unterscheidet man zwischen Technologie, Marktpreis, Regulatorik und Reputationsrisiken.  Transition risk for the real estate sector derive from  Regulatorik, policy changes related to the transition away from energy prices and CO2
taxes. 
Because Households with less energy-efficient homes may be more exposed to energy price shocks. This is because they need to consume more energy to heat their homes. Some households may also offset some of these rising energy costs by improving the energy-efficiency of their homes. While this reduces households’ exposure to energy price shocks, this also comes with financial costs.
\subsubsection{Physische Risiken}

\subsubsection{Abflussmenge bei Hochwasser}\label{sec:HQ}
Die gesetzlich vorgeschriebene Identifikation von Hochwasserrisikogebieten basiert auf dem Konzept des HQ\textsubscript{T\textsubscript{n}}-Ereignissen \autocite{WHG73}. Hierbei steht HQ für die Abflussmenge bei Hochwasser, während T\textsubscript{n} die statistische Jährlichkeit des Ereignisses repräsentiert. Beispielsweise bezeichnet ein HQ\textsubscript{100} ein statistisch einmal in 100 Jahren auftretendes Hochwasserereignis.
In Bayern erfolgt eine detailliertere Klassifizierung von Hochwasserereignissen anhand ihrer statistischen Häufigkeit \autocite{BayLfU2019}:
\begin{itemize}
\item HQ\textsubscript{häufig}: Ein Hochwasser, das im Mittel alle 5 bis 20 Jahre auftritt und als "häufiges Hochwasser" bezeichnet wird.
\item HQ\textsubscript{100}: Ein Ereignis, das statistisch einmal in 100 Jahren zu erwarten ist.
\item HQ\textsubscript{extrem}: Ein sehr seltenes Extremhochwasser, das zu deutlich höheren Wasserständen als ein HQ\textsubscript{100} führen kann.
\end{itemize}
Diese Kategorisierung ermöglicht eine differenzierte Bewertung von Hochwasserrisiken und unterstützt die Planung von Schutzmaßnahmen.
