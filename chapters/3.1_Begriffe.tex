
\subsection{Wesentliche Begriffe}
\subsubsection{Beleihungsauslauf}
Der Beleihungsauslauf, auch als Loan-to-Value-Ratio bekannt, ist eine zentrale Kennzahl in der Immobilienfinanzierung \parencite{BelWertV_3}. Er wird definiert als:
\begin{equation}
    \text{Beleihungsauslauf} = \frac{\text{Darlehensbetrag}}{\text{Beleihungswert}}
    \label{eq:ltv}
\end{equation}

\noindent wobei:
\begin{itemize}
    \item Der Darlehensbetrag die Höhe des Kredits repräsentiert.
    \item Der Beleihungswert den langfristig realisierbaren Wert der Immobilie darstellt, unabhängig von kurzfristigen Marktschwankungen.
\end{itemize}

Ein niedrigerer Beleihungsauslauf impliziert ein geringeres Ausfallrisiko für den Kreditgeber. Konsequenterweise führt ein höherer Beleihungsauslauf in der Regel zu ungünstigeren Darlehenskonditionen für den Kreditnehmer.

Zur Veranschaulichung des Konzepts wird ein Beispiel einer Immobilienfinanzierung betrachtet:
Eine Immobilie wird von einem Sachverständigen mit einem Beleihungswert von 500.000 € eingeschätzt. Von der Person, die das Haus kaufen möchte, wird ein Darlehen in Höhe von 275.000 € beantragt.
Der Beleihungsauslauf wird gemäß Gleichung \ref{eq:ltv} folgendermaßen berechnet:
\begin{equation}
    \text{Beleihungsauslauf} = \frac{\text{Darlehensbetrag}}{\text{Beleihungswert}} = \frac{275.000 \mbox{\texteuro}}{500.000 \mbox{\texteuro}} = 0,55 = 55\%
\end{equation}
Es wird festgestellt, dass der Beleihungsauslauf in diesem Fall 55\% beträgt. Dies bedeutet, dass 55\% des von der Bank anerkannten Immobilienwertes durch das Darlehen finanziert werden. Die restlichen 45\% müssen durch Eigenkapital oder andere Mittel aufgebracht werden.
Ein Beleihungsauslauf von 55\% wird allgemein als vorsichtig eingestuft. Es kann angenommen werden, dass bei einem solchen Wert günstigere Darlehenskonditionen gewährt werden, da das Ausfallrisiko für die Bank als relativ gering betrachtet wird.
\subsubsection{Transitionsrisiken}
Transitionsrisiken bezeichnen das Risiko finanzieller Verluste für Institutionen, die aus dem Anpassungsprozess hin zu einer Wirtschaft mit weniger CO2-Emissionen und einer umweltfreundlicheren Ökonomie entstehen \parencite{ecb2020climate}. Dieses Konzept, das in den Bereichen ESG, Klimawandel, Wirtschaft und Finanzen breite Anwendung findet, umfasst vier Hauptkategorien: Technologie-, Marktpreis-, Regulierungs- und Reputationsrisiken.

Besonders exponiert gegenüber diesen Risiken ist der Immobiliensektor, der für etwa 40\% der globalen Treibhausgasemissionen verantwortlich zeichnet \parencite{unepfi2023realestate} und somit vor erheblichen Anpassungsherausforderungen steht.
Für diesen Sektorentstehen Transitionsrisiken vorwiegend durch regulatorische Änderungen, insbesondere durch neue Gesetze zu Energiepreisen und CO2-Steuern. Diese Entwicklungen zwingen den Sektor zu weitreichenden Anpassungen, um wettbewerbsfähig zu bleiben und gleichzeitig Nachhaltigkeitsziele zu erreichen.
\subsubsection{Physische Risiken}
Unter physischen Risiken versteht man Gefahren, die aus natürlichen Ereignissen oder Umweltbedingungen entstehen und negative Konsequenzen für Gesellschaft, Wirtschaft und Ökosysteme nach sich ziehen können \parencite{greenvisionsolutions_transitorische_2024}. Diese Risiken werden typischerweise in akute Risiken, die durch plötzliche extreme Ereignisse wie Überschwemmungen oder Stürme hervorgerufen werden, und chronische Risiken, die sich aus langfristigen klimatischen Veränderungen wie dem Meeresspiegelanstieg, Wasserstress, Biodiversitätsverlust und Ressourcenknappheit ergeben, unterteilt \parencite{dnb2019values}.

\subsubsection{Klimaszenarien des NGFS}
Das \ac{NGFS} hat 72 verschiedene Klimaszenarien entwickelt. In der vorliegenden Arbeit wird die Anzahl der Szenarien auf die wichtigsten sechs Einzelszenarien begrenzt. Diese sechs Szenarien decken das vernünftigste Spektrum möglicher Entwicklungen ab. Die Szenarien verteilen sich auf verschiedene Quadranten eines Rahmenwerks, das sowohl das Ausmaß der physischen Risiken als auch den Übergangspfad innerhalb des jeweiligen Klimaszenarios berücksichtigt. Eine Abbildung, die die verschiedenen Quadranten veranschaulicht, sowie eine Beschreibung dieser Quadranten finden sich in Abbildung 2.2 und Tabelle 2.1.
Die Szenarien reichen von geordneten Übergängen mit geringen Risiken bis hin zu ungeordneten Szenarien mit hohen Risiken. Die Verteilung der sechs Hauptszenarien, die vom NGFS vorgeschlagen wurden, ist in der genannten Abbildung dargestellt.
Für die vorliegende Untersuchung sind insbesondere die Szenarien "Orderly", "Disorderly" und "Hot House World" (HHW) für die Jahre 2030, 2040 und 2050 von Bedeutung. Diese Auswahl basiert auf dem Klimarisiko-Stresstest der Europäischen Zentralbank (EZB), der einen Bottom-up-Ansatz verfolgt und sowohl Transitions- als auch physische Risiken einbezieht. Die EZB stellt ein Basisszenario sowie sechs Klimawandelszenarien bereit, die auf Phase II der NGFS-Modelle basieren. Diese Szenarien bilden die Grundlage für die Analyse der potenziellen Auswirkungen verschiedener Klimapolitiken auf den Immobiliensektor über einen längeren Zeitraum und ermöglichen eine detaillierte Untersuchung der Bruttowertschöpfung in diesem Sektor.
\subsubsection{Abflussmenge bei Hochwasser}\label{sec:HQ}
Die gesetzlich vorgeschriebene Identifikation von Hochwasserrisikogebieten basiert auf dem Konzept des HQ\textsubscript{T\textsubscript{n}}-Ereignissen \autocite{WHG73}. Hierbei steht HQ für die Abflussmenge bei Hochwasser, während T\textsubscript{n} die statistische Jährlichkeit des Ereignisses repräsentiert. Beispielsweise bezeichnet ein HQ\textsubscript{100} ein statistisch einmal in 100 Jahren auftretendes Hochwasserereignis.
In Bayern erfolgt eine detailliertere Klassifizierung von Hochwasserereignissen anhand ihrer statistischen Häufigkeit \autocite{BayLfU2019}:
\begin{itemize}
\item HQ\textsubscript{häufig}: Ein Hochwasser, das im Mittel alle 5 bis 20 Jahre auftritt und als "häufiges Hochwasser" bezeichnet wird.
\item HQ\textsubscript{100}: Ein Ereignis, das statistisch einmal in 100 Jahren zu erwarten ist.
\item HQ\textsubscript{extrem}: Ein sehr seltenes Extremhochwasser, das zu deutlich höheren Wasserständen als ein HQ\textsubscript{100} führen kann.
\end{itemize}
Diese Kategorisierung ermöglicht eine differenzierte Bewertung von Hochwasserrisiken und unterstützt die Planung von Schutzmaßnahmen.
