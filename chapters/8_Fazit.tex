\section{Fazit}\label{kap:7}
Zu Beginn der Forschung wurde folgende Frage gestellt: \enquote{Welche Faktoren bestimmen die Größe und Zusammensetzung eines Wohnimmobilienkreditportfolios und welche Datenquellen sind erforderlich?}  Die Antwort ergab sich aus vier Teilaspekten: Erstens wurde eine repräsentative Stichprobengröße anhand von Daten der Münchener Hypothekenbank und der Cochran-Formel ermittelt, wobei die tatsächliche Beleihungsauslaufverteilung dem Bankbericht entnommen wurde. Zweitens erfolgte die Verteilung der Darlehen basierend auf der Bevölkerungsdichte, mit Koordinaten aus Bevölkerungsverteilungskarten des Statistischen Bundesamtes. Hierbei wurden Überschwemmungsgebiete, Geländehöhe und Überflutungstiefen mittels Open-Data-Karten von \textcite{suche_postleitzahl} dem Bayerischen Landesamt für Umwelt und der Bayerischen Vermessungsverwaltung identifiziert. Drittens wurden Quadratmeterpreise für jedes Darlehen anhand von Daten der Bayerischen Landesbodenkreditanstalt bestimmt. Viertens wurden vollständige Risikoparameter unter Berücksichtigung realistischer Grenzen für Wohnfläche und Beleihungsauslauf erstellt. Die resultierende Datenverteilung spiegelt die Bevölkerungsdichte wider und zeigt die Verteilung in Hochwasserrisikogebieten. Die Ergebnisse in der Tabelle \ref{tab:hypothekenuberblick} bestätigen die erfolgreiche Erstellung eines repräsentativen Portfolios.

Die zweite Frage \enquote{Welche potenziellen Gefahren könnten physische Klimaereignisse in Bayern in der Zukunft darstellen, und wie lassen sich diese Risiken finanziell für die Besicherung des Hypothekenportfolios quantifizieren?} lässt sich wie folgt beantworten: Der \ac{EZB}-Stresstest hat die Vulnerabilität Deutschlands gegenüber Hochwasserereignissen verdeutlicht. Basierend auf der Schadensfunktion von Huizinga et al. und den Basel-III-Vorgaben für Risikogewichte lassen sich die Überschwemmungsschäden und relevante Darlehensportfolio-Parameter quantifizieren.  Die Differenz zwischen der Schadenquote von betroffenen Immobilien (8\%) und der des Gesamtportfolios (0,03\%) sowie die Einbeziehung der \ac{AEP} in die jährliche Schadensprognose (0,2\%) verdeutlichen die Notwendigkeit einer differenzierten Risikobetrachtung. Die Ergebnisse zeigen, dass Hochwasserrisiken nicht gleichmäßig über alle Immobilien verteilt sind. Sie konzentrieren sich auf bestimmte geografische Gebiete. Diese Erkenntnisse sind für die Entwicklung präziser Risikomodelle, Versicherungsstrategien und Präventionsmaßnahmen im Immobiliensektor relevant.

Zur Beantwortung der dritten Frage \enquote{Wie werden sich die Energiepreise verändern und wie beeinflussen diese Risiken die finanzielle Quantifizierung von Hypothekenportfolios?} wurden Energiepreisdaten des NGFS verwendet und mittels einer von \textcite{tergerman} entwickelten Gleichung der Wertverlust energieineffizienter Immobilien in verschiedenen Energieszenarien berechnet. Die Ergebnisse zeigen eine signifikante Korrelation zwischen Energiepreisanstieg und Immobilienwertverlust, besonders bei energieineffizienten Objekten. Das Szenario Aktuelle Richtlinien erweist sich als am stabilsten, während Netto-Null die stärksten negativen Auswirkungen aufzeigt, was die Notwendigkeit verbesserter Energieeffizienz unterstreicht. Die Szenarien Ungeordnet und Unter 2°C präsentieren mittlere Risikoniveaus und verdeutlichen den Einfluss der Klimapolitik auf den Immobilienmarkt.

Diese Arbeit hat eine detaillierte geospatiale Analyse für Immobilien in Bayern durchgeführt. Sie hat erfolgreich eine Methode zur Quantifizierung entwickelt und ein realistisches Portfolio erstellt, um die Auswirkungen des Klimarisikos auf Hypothekendarlehen zu bewerten.  Bisher gibt es keine Studie, die sowohl physische als auch Übergangsrisiken untersucht, um einen Vergleich anzustellen. Bisherige Forschungen konzentrierten sich meist auf einen dieser Faktoren. In Bezug auf physische Risiken zeigte die Studie von \textcite{moore2022flood}, dass die Hauspreise in den am stärksten überschwemmten Regionen sanken. Da diese Untersuchung jedoch deutschlandweit durchgeführt wurde und einen anderen Immobiliendatensatz verwendete, ist ein direkter Vergleich nicht möglich. Hinsichtlich der Übergangsrisiken zeigt ein Vergleich mit der Studie von \textcite{tergerman} ähnliche Trends bei der Immobilienwertminderung. Sowohl in der genannten Studie als auch in dieser Arbeit erleiden energetisch ineffiziente Immobilien (wie Klasse G und H) die größten Wertverluste, während effizientere Gebäude geringere Verluste verzeichnen. In beiden Studien zeigt das Netto-Null-Szenario die gravierendsten Wertverluste, gefolgt von den Szenarien Ungeordnet und Unter 2°C. Das Szenario Aktuelle Richtlinie erweist sich als das stabilste mit den geringsten Wertverlusten. Allerdings zeigt die Studie von \textcite{tergerman} einen Wertverlust von 45\% für Immobilien der Klasse H im Netto-Null-Szenario, während die vorliegende Untersuchung einen Rückgang von etwa 40\% bis 2050 prognostiziert. Diese Abweichung ist plausibel, da die genannten Studie globale NGFS-Daten für Energiepreise verwendete, während diese Arbeit sich auf spezifische Daten für Deutschland konzentrierte.

Eine Einschränkung dieser Arbeit liegt in der Datenbasis. Aus Datenschutzgründen konnten keine direkten, detaillierten Wohnimmobiliendaten von Bankkunden verwendet werden. Ein alternativer Datensatz von ImmobilienScout24, bereitgestellt durch das RWI - Leibniz-Institut für Wirtschaftsforschung, wurde in Betracht gezogen. Dieser war jedoch nicht öffentlich zugänglich. Eine Anfrage beim RWI - Leibniz-Institut ergab, dass für die Nutzung ein Datennutzungsvertrag zwischen den Institutionen erforderlich wäre. Aufgrund zeitlicher Beschränkungen wurde diese Option nicht weiter verfolgt.
Zudem hätte dieser Datensatz nur die Quantifizierung des Übergangsrisikos ermöglicht, da wichtige Hypothekendaten wie der Beleihungsauslauf fehlten. Die Datenverteilung basierend auf der Bevölkerungsdichte erfüllte Datenschutzanforderungen. Mangels genauer Geodaten für Wohngebiete konnten Datenpunkte an unplausiblen Orten wie Straßen, Parks oder Flussufern platziert werden.
Bei der räumlichen Analyse gab es Herausforderungen mit den verwendeten Koordinatensystemen. Die Hochwasserkarten des Bayerischen Landesamts für Umwelt nutzen das ETRS89-System, während die Hypothekendaten im EPSG:3035-System vorlagen. ETRS89 wird oft von Behörden für präzise geografische Positionsbestimmungen verwendet. EPSG:3035 eignet sich besser für wirtschaftliche Analysen und Flächenberechnungen. Die Verwendung unterschiedlicher Systeme spiegelt verschiedene Anforderungen wider, kann aber bei der Konvertierung zu Ungenauigkeiten führen.

Für zukünftige Forschungen ergeben sich mehrere Möglichkeiten. Ein wichtiges Ziel wäre die Identifikation eines geeigneten Datensatzes, der sowohl physische als auch Übergangsrisiken bewerten kann. Dies würde eine umfassendere Analyse ermöglichen. Zudem könnte die Integration einer detaillierten Karte von Wohngebieten die Platzierung der Datenpunkte verbessern. Dadurch würde die räumliche Genauigkeit der Analyse erhöht. Eine weitere vielversprechende Möglichkeit wäre die Gestaltung einer leicht bedienbaren Webanwendung. Dies könnte sowohl für Personen, die ein Haus kaufen möchten, als auch für Banken nützlich sein. Nach Eingabe relevanter Daten könnte das Tool zwei wichtige Aspekte berechnen: Erstens den potenziellen Schaden an Immobilien im Falle von Naturkatastrophen. Zweitens die möglichen Veränderungen des Immobilienwertes basierend auf Energiepreisschwankungen. Ein derartiges Werkzeug würde die Umsetzung der Forschungsergebnisse in die Praxis unterstützen und könnte als nützliche Unterstützung für Entscheidungsprozesse dienen.