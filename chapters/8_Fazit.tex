\section{Fazit}
Zu Beginn der Forschung wurde folgende Frage gestellt:
Welche Faktoren bestimmen die Größe und Zusammensetzung eines Wohnimmobilienkreditportfolios und welche Datenquellen sind erforderlich?

Die Antwort ergab sich aus vier Teilaspekten: Erstens wurde eine repräsentative Stichprobengröße anhand von Daten der Münchener Hypothekenbank und der Cochran-Formel ermittelt, wobei die tatsächliche Beleihungsauslaufverteilung dem Bankbericht entnommen wurde. Zweitens erfolgte die Verteilung der Darlehen basierend auf der Bevölkerungsdichte, mit Koordinaten aus Bevölkerungsverteilungskarten des Statistischen Bundesamtes. Hierbei wurden Überschwemmungsgebiete, Geländehöhe und Überflutungstiefen mittels Open-Data-Karten von Schowochow, M. (2023), dem Bayerischen Landesamt für Umwelt und der Bayerischen Vermessungsverwaltung identifiziert. Drittens wurden Quadratmeterpreise für jedes Darlehen anhand von Daten der Bayerischen Landesbodenkreditanstalt bestimmt. Viertens wurden vollständige Risikoparameter unter Berücksichtigung realistischer Grenzen für Wohnfläche und Beleihungsauslauf erstellt. Die resultierende Datenverteilung spiegelt die Bevölkerungsdichte wider und zeigt die Verteilung in Hochwasserrisikogebieten. Die Ergebnisse in der Tabelle \ref{tab:hypothekenuberblick} bestätigen die erfolgreiche Erstellung eines repräsentativen Portfolios.

Die zweite Frage "Welche potenziellen Gefahren könnten physische Klimaereignisse in Bayern in der Zukunft darstellen, und wie lassen sich diese Risiken finanziell für die Besicherung des Hypothekenportfolios quantifizieren?" lässt sich wie folgt beantworten: Der EZB-Stresstest hat die Vulnerabilität Deutschlands gegenüber Hochwasserereignissen verdeutlicht. Basierend auf der Schadensfunktion von Huizinga et al. und den Basel-III-Vorgaben für Risikogewichte lassen sich die Überschwemmungsschäden und relevante Darlehensportfolio-Parameter quantifizieren.  Die Differenz zwischen der Schadenquote von betroffenen Immobilien (8,0\%) und der des Gesamtportfolios (0,03\%) sowie die Einbeziehung der AEP in die jährliche Schadensprognose (0,2\%) verdeutlichen die Notwendigkeit einer differenzierten Risikobetrachtung. Die geringe Anzahl betroffener Immobilien impliziert eine hohe Selektivität der Hochwasserexposition. Diese Erkenntnisse sind für die Entwicklung präziser Risikomodelle, Versicherungsstrategien und Präventionsmaßnahmen im Immobiliensektor relevant

Zur Beantwortung der dritten Frage "Wie werden sich die Energiepreise verändern und wie beeinflussen diese Risiken die finanzielle Quantifizierung von Hypothekenportfolios?" wurden Energiepreisdaten des NGFS verwendet und mittels einer von X entwickelten Gleichung der Wertverlust energieineffizienter Immobilien in verschiedenen Energieszenarien berechnet. Die Ergebnisse zeigen eine signifikante Korrelation zwischen Energiepreisanstieg und Immobilienwertverlust, besonders bei energieineffizienten Objekten. Das Szenario "Aktuelle Richtlinien" erweist sich als am stabilsten, während "Netto-Null" die stärksten negativen Auswirkungen aufzeigt, was die Notwendigkeit verbesserter Energieeffizienz unterstreicht. Die Szenarien "Ungeordnet" und "Unter 2°C" präsentieren mittlere Risikoniveaus und verdeutlichen den Einfluss der Klimapolitik auf den Immobilienmarkt.


Die Kartierungen des Bayerischen Landesamts für Umwelt bieten eine fundierte Grundlage für die regionale Risikoanalyse. Diese für die Einzugsgebiete von Donau, Rhein und Elbe entwickelten Karten ermöglichen eine präzise Einschätzung der Hochwasserrisiken in Bayern \parencite{LfU_Bayern}. Sie visualisieren detailliert die Hochwassergefährdung, potenziell betroffene Landnutzungen und historische Hochwasserereignisse.
Diese Daten liegen im ETRS89-Koordinatensystem vor, während die Hypothekengeodaten das EPSG:3035-System nutzen. Obwohl beide Systeme für Europa konzipiert sind, dienen sie unterschiedlichen Zwecken:
ETRS89 wird häufig von Behörden und für Vermessungsarbeiten genutzt. Es bietet eine hohe Genauigkeit für geografische Positionsbestimmungen. Das Bayerische Landesamt verwendet es, da es präzise Ortsangaben für Hochwasserrisiken ermöglicht.
EPSG:3035 hingegen wird oft für statistische und wirtschaftliche Analysen eingesetzt. Es eignet sich besonders gut für Flächenberechnungen und Distanzmessungen. Daher wird es für die Hypothekendaten bevorzugt, die oft räumliche Wirtschaftsanalysen erfordern.
Die Verwendung unterschiedlicher Systeme spiegelt also die verschiedenen Anforderungen der Datenerhebung und -nutzung wider. Behörden bevorzugen oft traditionelle geografische Koordinaten, während wirtschaftliche Analysen projektionsbezogene Systeme nutzen.
Zur Herstellung der Datenkohärenz erfolgt eine Koordinatentransformation in das EPSG:3035-System. Diese Umwandlung ermöglicht die Integration beider Datensätze für eine umfassende Analyse. Sie gewährleistet, dass Hochwasserrisiken und Hypothekendaten präzise überlagert und gemeinsam ausgewertet werden können.

\ref{fig:schadenereignis_jahr} 