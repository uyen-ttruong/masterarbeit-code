\section{Daten und Portfolio-Vorbereitung}

\begin{sloppypar}
\hyphenpenalty=10
\tolerance=1000
\emergencystretch=\hsize
Im Rahmen dieses Kapitels wird die Methodik zur Generierung eines repräsentativen Musterportfolios erläutert, das als Grundlage für die Prognose klimabedingter Schäden dient. Trotz des umfangreichen Datenbestands der Finanzinstitute über ihre Kredit\-engagements hat die limitierte Zugänglichkeit zu detaillierten Datensätzen bisher umfassende empirische Analysen der Kreditrisiken eingeschränkt. Zur Überwindung dieser Limitation wird ein Musterportfolio konstruiert, das eine fundierte Approximation der erwarteten Verluste aus Wohn\-immobilien\-krediten ermöglicht. Die Quantifizierung essenzieller Risikoparameter basiert primär auf dem Geschäftsbericht der Münchener Hypothekenbank (\cite{MuenchenerHyp2022}), ergänzt durch ausdifferenzierte Datensätze zur Distribution von Energie\-effizienz\-klassen und regionalen Verteilung von Wohneinheiten. Diese Datenaggregation bildet die Basis für eine detaillierte Analyse der Anfälligkeit verschiedener Immobilienarten und Standorte gegenüber umweltbedingten Wert\-änderungen, physischen Risiken sowie Transitions\-risiken.

Zum Stichtag 31.12.2023 belief sich der ausstehende Bestand an Wohnimmobilienfinanzierungen im Portfolio der Münchener Hypothekenbank in Bayern auf 8.921.489.311,€, wobei die durchschnittsgröße der Darlehen Wohnimmobilien circa 163.700,€ betrug. Zur Ermittlung der Anzahl der Darlehen im Portfolio wird zunächst der Gesamtbestand durch die durchschnittsgröße der Darlehen dividiert, was auf etwa 54.500 Darlehen schließen lässt.
Unter Anwendung der Gleichung \ref{eq:cochran} zur Berechnung der erforderlichen Stichprobengröße für das theoretische Szenario einer unendlichen Anzahl von Immobilien im Portfolio ergibt sich bei einem Konfidenzintervall von 99\% und einer Fehlermarge von 2\% ein notwendiger Stichprobenumfang von 4.147 Datenpunkten. Die in Gleichung \ref{eq:finite_population} präsentierte Formulierung für endliche Populationen führt jedoch zu einer Reduktion auf 3.853 Darlehen als erforderliche Stichprobengröße für Portfolios mit mehr als 54.500 Elementen.

\begin{table}[htbp]
    \centering
    \caption{Verteilung des Beleihungsauslaufs im Wohnimmobilienportfolio der Münchener Hypothekenbank zum 31.12.2023}
    \label{tab:beleihungsauslauf2023}
    \begin{tabular}{lcccccc}
    \toprule[0.5pt]
    Beleihungsauslauf & bis 60\% & $>$60-70\% & $>$70-80\% & $>$80-90\% & $>$90-100\% & $>$100\% \\
    Prozentanteil & 39,2\% & 15,0\% & 16,4\% & 10,2\% & 8,2\% & 11,0\% \\
    \bottomrule[1.5pt]
    \end{tabular}
\end{table}

"Determinants of the Demand for Housing in Germany" (Flaig, G., & Steiner, V. 1993): This older paper analyzes the factors influencing housing demand in Germany. It finds that population growth is a key driver of housing demand. Since population growth often leads to increased population density, this indirectly supports the hypothesis.

\end{sloppypar}

