% cSpell:disable
\section{Daten und Portfolio-Vorbereitung}

\begin{sloppypar}
\hyphenpenalty=10
\tolerance=1000
\emergencystretch=\hsize
Im Rahmen dieses Kapitels wird die Methodik zur Generierung eines repräsentativen Musterportfolios erläutert, das als Grundlage für die Prognose klimabedingter Schäden dient. Trotz des umfangreichen Datenbestands der Finanzinstitute über ihre Kredit\-engagements hat die limitierte Zugänglichkeit zu detaillierten Datensätzen bisher umfassende empirische Analysen der Kreditrisiken eingeschränkt. Zur Überwindung dieser Limitation wird ein Musterportfolio konstruiert, das eine fundierte Approximation der erwarteten Verluste aus Wohn\-immobilien\-krediten ermöglicht. Die Quantifizierung essenzieller Risikoparameter basiert primär auf dem Geschäftsbericht der Münchener Hypothekenbank (\cite{MuenchenerHyp2022}), ergänzt durch ausdifferenzierte Datensätze zur Distribution von Energie\-effizienz\-klassen und regionalen Verteilung von Wohneinheiten. Diese Datenaggregation bildet die Grundlage für eine detaillierte Analyse der Anfälligkeit verschiedener Immobilienarten und Standorte gegenüber physischen Risiken sowie Transitionsrisiken.
\end{sloppypar}
\newpage
toi la toi
