\section{Daten und Portfolio-Vorbereitung}

Im Rahmen dieses Kapitels wird die Methodik zur Generierung eines repräsentativen Musterportfolios erläutert, welches als Grundlage für die Prognose klimabedingter Schäden dient. Finanzinstitute, insbesondere Banken, verfügen üblicherweise über ein umfangreiches Spektrum an granularen Daten bezüglich ihrer Kreditengagements. Diese umfassen präzise Angaben zu geografischen Standorten, Immobilientypologien, Flächenmaßen und Energieeffizienzklassifizierungen. Darüber hinaus beinhalten diese Daten auch Informationen zu den angewandten Kreditvergabestandards.

Die limitierte Zugänglichkeit zu detaillierten Datensätzen hat bisher die Durchführung umfassender empirischer Analysen der Kreditrisiken in den Immobilienkreditportfolios deutscher Finanzinstitute signifikant eingeschränkt. Zur Überwindung dieser Limitation wird die Konstruktion eines repräsentativen Musterportfolios vorgeschlagen. Dieses ermöglicht eine fundierte Approximation der erwarteten Verluste aus Wohnimmobilienkrediten , welche für Kreditinstitute von erheblicher Relevanz sind.

Die Quantifizierung essenzieller Risikoparameter wie Beleihungsauslauf, Ausfallwahrscheinlichkeit, Verlustquote bei Ausfall und Ausfallkredithöhe basiert primär auf dem Geschäftsbericht 2023 der MünchenerHypothekenbank. Zur Evaluation klimabedingter Risiken werden zusätzlich granulare Daten zur Distribution von Energieeffizienzklassen im deutschen Immobilienbestand sowie zur regionalen Verteilung von Wohneinheiten und Eigenheimen in Bayern integriert. Diese Synthese ermöglicht nicht nur ein repräsentatives Musterportfolio, sondern schafft auch die Grundlage für eine differenzierte Analyse der Vulnerabilität verschiedener Immobilientypen und Standorte gegenüber klimainduzierten Wertveränderungen und physischen Risiken.