\section{Einleitung}
\subsection{Motivation}

Das Eigenheim ist für viele Menschen nicht nur ein Dach über dem Kopf, sondern auch der bedeutendste finanzielle Meilenstein ihres Lebens. Nach Gesundheit und Bildung stellt der Erwerb einer Immobilie oft die größte Investition dar, die ein Mensch tätigt. Dieser Schritt erfordert in den meisten Fällen die Unterstützung durch Banken in Form von Immobilienkrediten. Tatsächlich machen diese Kredite einen beträchtlichen Teil des gesamten Kreditgeschäfts von Banken aus. Laut einer Studie der Deutschen Bundesbank (2023) entfielen im Jahr 2022 etwa 40% aller ausstehenden Kredite an inländische Unternehmen und Privatpersonen auf Wohnungsbaukredite.
Die Bedeutung dieses Sektors wird noch deutlicher, wenn man spezialisierte Institute wie Hypothekenbanken betrachtet, die sich ausschließlich auf die Immobilienfinanzierung konzentrieren. Angesichts der enormen finanziellen Verpflichtungen, die sowohl Kreditnehmer als auch Kreditgeber eingehen, rückt die Risikoanalyse in den Mittelpunkt des Interesses. Diese Risikobewertung gewinnt in Zeiten des Klimawandels zusätzlich an Komplexität und Dringlichkeit.
Im Kontext der ESG-Kriterien (Environmental, Social, Governance), die zunehmend als Maßstab für nachhaltige Unternehmensführung herangezogen werden, fällt die Bewertung von Immobilienrisiken primär unter den Umweltaspekt (E). Klimarisiken lassen sich hier besonders gut quantifizieren, da sie direkte und messbare Auswirkungen auf den Wert und die Sicherheit von Immobilien haben können. Eine Studie des Umweltbundesamtes (2021) zeigt, dass bis 2050 etwa 12% der Gebäude in Deutschland von klimabedingten Risiken wie Überschwemmungen oder Stürmen betroffen sein könnten, was potenzielle Schäden in Milliardenhöhe bedeutet.
Sowohl Käufer als auch Kreditgeber haben ein vitales Interesse an nachhaltigen und klimaresilienten Investitionen. Käufer möchten sicherstellen, dass ihr Eigenheim nicht in einer gefährdeten Zone liegt und gleichzeitig energieeffizient ist, um Betriebskosten zu senken und den CO2-Fußabdruck zu reduzieren. Banken ihrerseits müssen die langfristige Werthaltigkeit ihrer Sicherheiten gewährleisten. Dieser Aspekt gewinnt zusätzlich an Gewicht durch die Richtlinien der Europäischen Zentralbank (EZB), die Banken explizit auffordert, sich mit den Auswirkungen des Klimawandels auseinanderzusetzen und diese in ihre Risikomodelle zu integrieren.
Das Ziel dieser Arbeit ist es, eine umfassende Methodik zur Integration von Klimarisiken und geospatialer Analyse in die Bewertung von Immobilienportfolios zu entwickeln und anzuwenden. Im Einzelnen verfolgt die Studie drei Hauptziele:

Die Erstellung eines repräsentativen Wohnimmobilienkreditportfolios für Bayern unter Berücksichtigung relevanter Faktoren wie Bevölkerungsdichte, Überschwemmungsgebiete und Immobilienpreise. Hierbei sollen verschiedene Datenquellen identifiziert und genutzt werden, um ein möglichst realistisches Bild zu erzeugen.
Die Quantifizierung potenzieller Gefahren durch physische Klimaereignisse, insbesondere Hochwasser, für das erstellte Hypothekenportfolio in Bayern. Dabei sollen finanzielle Auswirkungen auf die Besicherung des Portfolios berechnet und eine differenzierte Risikobetrachtung vorgenommen werden.
Die Analyse der Auswirkungen sich verändernder Energiepreise auf die Wertentwicklung von Immobilien in verschiedenen Klimaszenarien. Hierbei soll insbesondere der Zusammenhang zwischen Energieeffizienz und potenziellem Wertverlust untersucht werden.

Durch die Kombination dieser Aspekte strebt die Arbeit an, sowohl physische als auch Übergangsrisiken in einem einheitlichen Rahmen zu betrachten und damit eine Lücke in der bestehenden Forschung zu schließen. Das Ergebnis soll eine robuste Methode zur Quantifizierung von Klimarisiken in Immobilienportfolios liefern, die Banken und Investoren bei der Entwicklung präziser Risikomodelle, Versicherungsstrategien und Präventionsmaßnahmen unterstützen kann.
Vor diesem Hintergrund widmet sich diese Arbeit drei zentralen Forschungsfragen:

Welche Faktoren sind bei der Bestimmung der Größe und Zusammensetzung eines Wohnimmobilienkreditportfolios zu berücksichtigen, und welche Datenquellen sind zur Erhebung der erforderlichen Informationen notwendig?
Welche potenziellen Gefahren könnten physische Klimaereignisse in Bayern in der Zukunft darstellen, und wie lassen sich diese Risiken finanziell für die Besicherung des Hypothekenportfolios quantifizieren?
Wie werden sich die Energiepreise verändern und wie beeinflussen diese Risiken die finanzielle Quantifizierung von Hypothekenportfolios?

Diese Fragen sind von großer Relevanz, da die Integration von Klimarisiken in die Immobilienbewertung nicht nur eine regulatorische Anforderung darstellt, sondern auch ein entscheidender Faktor für die langfristige Stabilität des Finanzsektors ist. Eine Studie der Bank für Internationalen Zahlungsausgleich (2021) schätzt, dass klimabedingte Risiken bis zu 25\% des Wertes von Immobilienportfolios beeinflussen können, wenn sie nicht angemessen berücksichtigt und gemanagt werden.
Die vorliegende Arbeit zielt darauf ab, einen Beitrag zur Entwicklung robuster Methoden für die Integration von Klimarisiken und geospatialen Analysen in die Bewertung von Immobilienportfolios zu leisten. Durch die Kombination von Klimamodellen, geographischen Informationssystemen und traditionellen Bewertungsmethoden soll ein ganzheitlicher Ansatz entwickelt werden, der es Banken und Investoren ermöglicht, fundierte und zukunftsorientierte Entscheidungen zu treffen. Besonderes Augenmerk wird dabei auf die spezifische Situation in Bayern gelegt, um regionale Besonderheiten und Risiken adäquat zu berücksichtigen.
Diese Arbeit schließt eine wichtige Forschungslücke, indem sie sowohl physische als auch Übergangsrisiken in einem einheitlichen Rahmen betrachtet. Bisher haben sich Studien meist auf einen dieser Faktoren konzentriert. Die hier entwickelte Methodik zur Quantifizierung von Klimarisiken in Immobilienportfolios soll Banken und Investoren bei der Entwicklung präziser Risikomodelle, Versicherungsstrategien und Präventionsmaßnahmen unterstützen und damit einen Beitrag zur Stabilität des Immobilien- und Finanzsektors im Kontext des Klimawandels leisten.

Der Erwerb einer Immobilie stellt für viele Menschen die größte finanzielle Investition ihres Lebens dar. Immobilienkredite machen einen beträchtlichen Teil des Bankgeschäfts aus, wie Daten der Deutschen Bundesbank (2023) belegen. In Zeiten des Klimawandels gewinnt die Risikoanalyse in diesem Sektor zunehmend an Bedeutung.
Im Rahmen der ESG-Kriterien fallen Immobilienrisiken primär unter den Umweltaspekt. Eine Studie des Umweltbundesamtes (2021) prognostiziert, dass bis 2050 etwa 12% der Gebäude in Deutschland von klimabedingten Risiken betroffen sein könnten. Sowohl Käufer als auch Kreditgeber haben daher ein vitales Interesse an klimaresilienten Investitionen. Die Europäische Zentralbank fordert Banken explizit auf, Klimawandelauswirkungen in ihre Risikomodelle zu integrieren.
Das Ziel dieser Arbeit ist die Entwicklung einer umfassenden Methodik zur Integration von Klimarisiken und geospatialer Analyse in die Bewertung von Immobilienportfolios. Dabei werden drei Hauptziele verfolgt:

Erstellung eines repräsentativen Wohnimmobilienkreditportfolios für Bayern.
Quantifizierung potenzieller Gefahren durch physische Klimaereignisse, insbesondere Hochwasser.
Analyse der Auswirkungen sich verändernder Energiepreise auf die Immobilienwertentwicklung in verschiedenen Klimaszenarien.

Die zentralen Forschungsfragen lauten:

Welche Faktoren und Datenquellen sind für die Bestimmung eines Wohnimmobilienkreditportfolios relevant?
Wie lassen sich physische Klimarisiken für Hypothekenportfolios in Bayern finanziell quantifizieren?
Wie beeinflussen Energiepreisänderungen die finanzielle Bewertung von Hypothekenportfolios?

Diese Arbeit schließt eine Forschungslücke, indem sie sowohl physische als auch Übergangsrisiken in einem einheitlichen Rahmen betrachtet. Die entwickelte Methodik soll Banken und Investoren bei der Entwicklung präziser Risikomodelle unterstützen und zur Stabilität des Immobilien- und Finanzsektors im Kontext des Klimawandels beitragen.
\subsection{Literaturüberblick }
Die Forschung zu Klimarisiken und deren Auswirkungen auf Immobilienwerte hat in den letzten Jahren erheblich an Bedeutung gewonnen. Grundlegende Arbeiten wie die von Schellekens und Toor (2019) haben die Notwendigkeit einer umfassenden Betrachtung von Nachhaltigkeitsrisiken im Finanzsektor aufgezeigt. Im Bereich der physischen Risiken lieferten Moore et al. (2022) wichtige Erkenntnisse über die Auswirkungen von Überschwemmungen auf den deutschen Immobilienmarkt, wobei sie einen signifikanten Wertrückgang in stark betroffenen Gebieten feststellten. Parallel dazu untersuchten Huizinga et al. (2017) die Schadensfunktionen bei Überschwemmungen für verschiedene europäische Regionen, was eine wichtige Grundlage für die Quantifizierung von Hochwasserrisiken bildet. Im Kontext der Transitionsrisiken haben Steege und Vogel (2021) eine wegweisende Studie zur Bewertung deutscher Wohnimmobilien unter verschiedenen NGFS-Klimaszenarien durchgeführt. Ihre Ergebnisse zeigen deutliche Wertverluste für energetisch ineffiziente Gebäude, insbesondere im Netto-Null-Szenario. Zurek (2022) ergänzte diese Perspektive durch die Analyse des Zusammenhangs zwischen lokalem Wirtschaftswachstum und Kreditrisiken im Immobiliensektor. Die Arbeit von Vanweddingen (2023) lieferte zudem wichtige methodische Ansätze zur Integration von Klimarisiken in Risikomodelle für Wohnimmobilienkredite. Trotz dieser bedeutenden Beiträge bleibt eine integrierte Betrachtung von physischen und Transitionsrisiken, insbesondere auf regionaler Ebene, eine Forschungslücke. Die vorliegende Arbeit zielt darauf ab, diese Lücke zu schließen, indem sie beide Risikoarten in einem einheitlichen Rahmen für den bayerischen Immobilienmarkt analysiert und quantifiziert.
\subsection{Aufbau der Arbeit}
Nach diesem einleitenden Kapitel folgt Kapitel 2, das den theoretischen Hintergrund dieser Arbeit darlegt. Zunächst werden wesentliche Begriffe und Konzepte wie Klimarisiko, physische Risiken und Transitionsrisiken erläutert. Anschließend werden die statistischen Grundlagen für die Analyse präsentiert. Kapitel 3 widmet sich der detaillierten Datenanalyse, beginnend mit Hypothekendaten, gefolgt von Geodaten zu Hypotheken und Hochwasser, Quadratmeterpreisen, Energieeffizienzklassen und Energiepreisen nach NGFS-Szenarien. In Kapitel 4 wird die Methodik ausführlich beschrieben, einschließlich des Datenaufbereitungsprozesses sowie der Quantifizierung physischer und Transitionsrisiken. Kapitel 5 präsentiert die Ergebnisse der Analyse, unterteilt in Portfoliodatensatz, physische Risiken und Transitionsrisiken. Die Diskussion in Kapitel 6 interpretiert die Ergebnisse und beleuchtet die Limitationen der Studie. Abschließend fasst das Fazit in Kapitel 7 die Haupterkenntnisse zusammen, erörtert die Implikationen für Praxis und Forschung und gibt einen Ausblick auf zukünftige Forschungsmöglichkeiten.