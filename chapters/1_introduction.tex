\section{Einleitung}
\subsection{Motivation}

In der heutigen Gesellschaft gelten Investitionen in Gesundheit und Bildung als primäre finanzielle Entscheidungen. Wohnbedürfnisse folgen in ihrer Wichtigkeit. Viele betrachten den Erwerb von Immobilien als wichtige Investition. Hierfür sind oft Bankkredite nötig.
Diese Kredite bilden einen erheblichen Teil des Bankgeschäfts, besonders bei Hypothekenbanken. Die hohen finanziellen Verpflichtungen erfordern sorgfältige Risikoanalysen. Der Klimawandel erhöht dabei die Komplexität der Bewertungen. Im Rahmen der ESG-Kriterien fällt die Immobilienrisikobewertung primär unter den Umweltaspekt.Klimarisiken sind hier gut quantifizierbar und beeinflussen direkt den Immobilienwert. 

Käufer und Kreditgeber haben Interesse an nachhaltigen und klimaresilienten Investitionen. Käufer wünschen sichere, energieeffiziente Häuser. Banken müssen die Werthaltigkeit ihrer Sicherheiten sicherstellen. Die EZB fordert zudem die Integration von Klimarisiken in Bankmodelle.

Einige Anbieter haben bereits Lösungen wie K.A.R.L.® entwickelt, um diesen Anforderungen zu entsprechen. Dieses Tool erstellt für jede gewünschte Adresse eine automatisierte Standortanalyse zu Naturgefahren wie Überschwemmung, Sturmflut, Sturm, Hagel und Starkregen, um die entsprechenden Naturkatastrophenrisiken zu bewerten. Allerdings ist es kostenpflichtig und fokussiert nur auf physische Risiken.

Diese Arbeit zielt daher ab, eine umfassende Methodik zur Integration von Klimarisiken und geospatialer Analyse in die Bewertung von Immobilienportfoliorisiken zu entwickeln und anzuwenden. Diese Vorgehensweise sollte sowohl von Hauskäufern als auch von Banken angewendet werden können.

Dabei werden drei zentralen Forschungsfragen verfolgt:

\begin{enumerate}
    \item Welche Faktoren und Datenquellen sind für die Bestimmung eines Wohnimmobilienkreditportfolios relevant?
    \item Wie lassen sich physische Klimarisiken für Hypothekenportfolios in Bayern finanziell quantifizieren?
    \item Wie beeinflussen Energiepreisänderungen die finanzielle Bewertung von Hypothekenportfolios?
    \end{enumerate}

\subsection{Literaturüberblick }
Der Literaturüberblick umfasst drei zentrale Studien. \textcite{moore2022flood} analysierten die Auswirkungen von Überschwemmungen auf den deutschen Immobilienmarkt und zeigten Wertrückgänge in betroffenen Gebieten auf. \text{huizinga2017global} entwickelten Schadensfunktionen für Überschwemmungen in Europa, die als Basis für die Quantifizierung von Hochwasserrisiken dienen. \textcite{tergerman} bewerteten deutsche Wohnimmobilien unter NGFS-Klimaszenarien und identifizierten signifikante Wertverluste für energetisch ineffiziente Gebäude. Diese Arbeiten bilden die Grundlage für die Analyse physischer und Transitionsrisiken in der vorliegenden Studie.

\subsection{Aufbau der Arbeit}

Nach diesem einleitenden Kapitel folgt Kapitel 2, das den theoretischen Hintergrund dieser Arbeit darlegt. Zunächst werden wesentliche Begriffe und Konzepte wie Klimarisiko, physische Risiken und Transitionsrisiken erläutert. Anschließend werden die statistischen Grundlagen für die Analyse präsentiert. Kapitel 3 analysiert Daten zu Hypotheken, Geodaten und Energiepreisen. Kapitel 4 beschreibt die Methodik zur Quantifizierung von Risiken. Kapitel 5 präsentiert die Ergebnisse für das Portfolio sowie physische und Transitionsrisiken. Die Diskussion in Kapitel 6 interpretiert die Ergebnisse. Abschließend fasst das Fazit in Kapitel 7 die Haupterkenntnisse zusammen, erörtert die Implikationen für Praxis und Forschung, beleuchtet die Limitationen der Studie. gibt einen Ausblick auf zukünftige Forschungsmöglichkeiten.