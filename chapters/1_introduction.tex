\section{Einleitung}\label{kap:1}
\subsection{Motivation}

In der heutigen Gesellschaft gelten Investitionen in Gesundheit und Bildung als primäre finanzielle Entscheidungen. Wohnbedürfnisse folgen in ihrer Wichtigkeit. Viele betrachten den Erwerb von Immobilien als wichtige Investition. Hierfür sind oft Bankkredite nötig.
Diese Kredite bilden einen erheblichen Teil des Bankgeschäfts, besonders bei Hypothekenbanken. Die hohen finanziellen Verpflichtungen erfordern sorgfältige Risikoanalysen. Der Klimawandel erhöht dabei die Komplexität der Bewertungen. Im Rahmen der \ac{ESG}-Kriterien fällt die Immobilienrisikobewertung primär unter den Umweltaspekt.Klimarisiken sind hier gut quantifizierbar und beeinflussen direkt den Immobilienwert. 

Käufer und Kreditgeber haben Interesse an nachhaltigen und klimaresilienten Investitionen. Käufer wünschen sichere, energieeffiziente Häuser. Banken müssen die Werthaltigkeit ihrer Sicherheiten sicherstellen. Die EZB fordert zudem die Integration von Klimarisiken in Bankmodelle.

Die Köln Asesekuranz Agentur hat die innovative Lösung K.A.R.L.® entwickelt, die von vielen Banken verwendet wird, um auf diese Anforderungen zu erfüllen. Dieses Tool erstellt für jede gewünschte Adresse eine automatisierte Standortanalyse zu Naturgefahren wie Überschwemmung, Sturmflut, Sturm, Hagel und Starkregen, um die entsprechenden Naturkatastrophenrisiken zu bewerten. Allerdings ist es kostenpflichtig und fokussiert nur auf physische Risiken.

Diese Arbeit zielt daher ab, eine umfassende Methodik zur Integration von Klimarisiken und geospatialer Analyse in die Bewertung von Immobilienportfoliorisiken zu entwickeln und anzuwenden. Diese Vorgehensweise sollte sowohl von Hauskäufern als auch von Banken angewendet werden können.

Dabei werden drei zentralen Forschungsfragen verfolgt:

\begin{enumerate}
    \item Welche Faktoren und Daten sind für die Bestimmung eines Wohnimmobilienkreditportfolios relevant?
    \item Wie lassen sich physische Klimarisiken für Hypothekenportfolios in Bayern finanziell quantifizieren?
    \item Wie beeinflussen Energiepreisänderungen die finanzielle Bewertung von Hypothekenportfolios?
    \end{enumerate}

\subsection{Literaturüberblick }
Der Literaturüberblick umfasst fünf zentrale Studien. \textcite{huizinga2017global} erstellten globale Hochwasserschadenfunktionen für verschiedene Vermögensklassen und Regionen. Diese Funktionen basieren auf Regressionsanalysen mit Baudaten aus verschiedenen Ländern. Sie bilden eine wichtige Grundlage für die Schätzung wirtschaftlicher Verluste durch Überschwemmungen.\\ \textcite{moore2022flood} untersuchten die Auswirkungen von Überschwemmungen auf den deutschen Immobilienmarkt. Ihre Studie zeigte signifikante Preisrückgänge in den am stärksten betroffenen Gebieten. Diese Ergebnisse unterstreichen die Bedeutung von Hochwasserrisiken für die Immobilienbewertung. Die \textcite{ECB2022ClimateStressTest} veröffentlichte Leitlinien zum Management von Klima- und Umweltrisiken für Banken. Diese Leitlinien fordern eine umfassende und zukunftsorientierte Bewertung physischer Risiken, einschließlich Hochwasserrisiken. Sie betonen die Notwendigkeit, diese Risiken in das Risikomanagement und die Geschäftsstrategie zu integrieren. \textcite{tergerman} führten eine detaillierte Analyse deutscher Wohnimmobilien unter verschiedenen Klimaszenarien durch. Ihre Studie nutzte Daten zu Energieeffizienz und Immobilienpreisen in Verbindung mit Klimaprojektionen. Sie kamen zu dem Schluss, dass energetisch ineffiziente Gebäude in Zukunft stark an Wert verlieren könnten, insbesondere in Szenarien mit strengen Klimaschutzmaßnahmen. \textcite{vanweddingen2023physicalrisk} entwickelte einen fortschrittlichen Prototyp zur Hochwasserrisikobewertung für Finanzinstitute. Dieser Ansatz kombiniert detaillierte Geodaten, Klimaszenarien und Schadensmodelle, um die Auswirkungen auf Immobilienkreditportfolios in europäischen Ländern zu quantifizieren. Diese Arbeiten bilden die Grundlage für die Analyse physischer und Transitionsrisiken in der vorliegenden Studie.
\subsection{Aufbau der Arbeit}

Die Arbeit gliedert sich in sieben zentrale Kapitel. Kapitel \ref{kap:1} bietet eine Einführung in die Thematik und formuliert die Motivation sowie die zentralen Forschungsfragen. Außerdem enthält es eine Zusammenfassung der Literatur und gibt einen Überblick über die Struktur der Arbeit.

Kapitel \ref{kap:2} stellt die theoretischen Grundlagen dar. Es werden wichtige Konzepte wie der Beleihungsauslauf, risikogewichtete Aktiva, sowie physische und transitorische Risiken im Kontext von Immobilienportfolios erklärt. Zudem wird auf verschiedene Klimaszenarien eingegangen, die als Grundlage für die Analyse der klimabedingten Risiken dienen.

Im dritten Kapitel \ref{sec:createportfolio} wird die Datenanalyse beschrieben. Hierbei werden die verwendeten Datensätze zu Hypotheken, Geodaten und Energiepreisen ausführlich erläutert. . Es wird zudem dargestellt, wie die Geodaten genutzt wurden, um physische Klimarisiken, wie Hochwassergefahren, zu modellieren und deren potenzielle Auswirkungen auf Immobilienwerte zu quantifizieren.

Kapitel \ref{kap:4} beschreibt die angewendete Methodik zur Quantifizierung der Klimarisiken. Der Schwerpunkt liegt auf der Datenaufbereitung und der Quantifizierung von Risiken. Zuerst wird der Prozess der Vorbereitung und Aggregation der Hypotheken- und Geodaten erläutert, um ein repräsentatives Portfolio zu erstellen. Anschließend wird die Methodik zur Quantifizierung physischer Risiken wie Hochwasserschäden und Transitionsrisiken durch Energiepreisschwankungen dargestellt. 
Im Kapitel \ref{kap:5} werden die Ergebnisse der Analyse präsentiert. Die Auswirkungen von Klimarisiken auf Immobilienportfolios werden detailliert dargestellt. Der Fokus liegt auf dem potenziellen Wertverlust von Immobilien. Zudem werden die Änderungen des Beleihungsauslaufs und der risikogewichteten Aktiva untersucht. Diese Veränderungen treten durch Hochwasserschäden oder Energiepreisänderungen auf.

Kapitel \ref{kap:6} umfasst die Diskussion der Ergebnisse. In diesem Kapitel werden die gewonnenen Erkenntnisse im Kontext der bestehenden Literatur diskutiert und die praktischen Implikationen für Banken und den Immobiliensektor beleuchtet.

Abschließend fasst das Fazit in Kapitel \ref{kap:7} die Haupterkenntnisse zusammen, erörtert die Implikationen für Praxis und Forschung, beleuchtet die Limitationen der Studie und gibt einen Ausblick auf zukünftige Forschungsmöglichkeiten