%karte discusstion
Die Kartierungen des Bayerischen Landesamts für Umwelt bieten eine fundierte Grundlage für die regionale Risikoanalyse. Diese für die Einzugsgebiete von Donau, Rhein und Elbe entwickelten Karten ermöglichen eine präzise Einschätzung der Hochwasserrisiken in Bayern \parencite{LfU_Bayern}. Sie visualisieren detailliert die Hochwassergefährdung, potenziell betroffene Landnutzungen und historische Hochwasserereignisse.
Diese Daten liegen im ETRS89-Koordinatensystem vor, während die Hypothekengeodaten das EPSG:3035-System nutzen. Obwohl beide Systeme für Europa konzipiert sind, dienen sie unterschiedlichen Zwecken:
ETRS89 wird häufig von Behörden und für Vermessungsarbeiten genutzt. Es bietet eine hohe Genauigkeit für geografische Positionsbestimmungen. Das Bayerische Landesamt verwendet es, da es präzise Ortsangaben für Hochwasserrisiken ermöglicht.
EPSG:3035 hingegen wird oft für statistische und wirtschaftliche Analysen eingesetzt. Es eignet sich besonders gut für Flächenberechnungen und Distanzmessungen. Daher wird es für die Hypothekendaten bevorzugt, die oft räumliche Wirtschaftsanalysen erfordern.
Die Verwendung unterschiedlicher Systeme spiegelt also die verschiedenen Anforderungen der Datenerhebung und -nutzung wider. Behörden bevorzugen oft traditionelle geografische Koordinaten, während wirtschaftliche Analysen projektionsbezogene Systeme nutzen.
Zur Herstellung der Datenkohärenz erfolgt eine Koordinatentransformation in das EPSG:3035-System. Diese Umwandlung ermöglicht die Integration beider Datensätze für eine umfassende Analyse. Sie gewährleistet, dass Hochwasserrisiken und Hypothekendaten präzise überlagert und gemeinsam ausgewertet werden können.

%portfolio
Tabelle \ref{tab:hypothekenuberblick} bietet einen umfassenden Überblick über die wesentlichen Finanzkennzahlen und statistischen Merkmale des analysierten Hypothekenportfolios. Die Darlehenbeträge, die im Mittel bei 163.700 € liegen, entsprechen genau der durchschnittlichen Darlehensgröße für Wohnimmobilien von Münchener Hypothekenbank. Die Quadratmeterpreise variieren erheblich, mit einem Durchschnitt von 4.727,16 € und einer Standardabweichung von 2.539,12 €, was die Diversität des Immobilienmarktes widerspiegelt. Die Wohnflächen im Portfolio reichen von 98,89 m² bis 250,00 m², mit einem Durchschnitt von 140,39 m² und einer Standardabweichung von 50,32 m². Diese Verteilung zeigt, dass das Portfolio verschiedene Wohnungsgrößen umfasst, wobei der Schwerpunkt auf mittelgroßen bis größeren Wohneinheiten liegt. Der durchschnittliche aktuelle Immobilienwert liegt bei 471.408,79 €. In Verbindung mit dem durchschnittlichen Beleihungsauslauf von 52\% lässt sich auf eine konservative Kreditvergabepraxis schließen. Dieses LTV-Verhältnis deutet nicht nur auf eine vorsichtige Risikoeinschätzung seitens der Bank hin, sondern auch auf eine solide Eigenkapitalbasis der Kreditnehmer. Die Spannbreite der Darlehensbeträge, von 27.395,06 € bis 508.290,10 €, zeigt zudem, dass das Portfolio sowohl kleinere als auch größere Finanzierungen umfasst, was zur Risikodiversifizierung beiträgt. Zusammenfassend lässt sich sagen, dass diese Daten ein ausgewogenes und realistisches Hypothekenportfolio repräsentieren,

%phyrisk
Aufgrund der Größe der \ac{DGM}-Daten (240 GB) erfolgte die Berechnung nur für 30 Punkte in Hochrisiko-, mittlerem und niedrigem Risikogebiet. Entsprechende Orts-, Gemeinde- und Landkreisdaten wurden geladen.
Im Hochrisikogebiet liegt ein einzelner Punkt in Haag an der Amper. Dort beträgt die maximale Überflutungstiefe 2,1 m, entsprechend einem Schadensfaktor von 0,16.
17 Datenpunkte befinden sich in Gebieten mittleren Risikos, verteilt auf verschiedene Orte.In der Kategorie mittleres Risiko gibt es Punkte mit einer Überschwemmungstiefe von 0. Dies ist durchaus plausibel. Innerhalb eines Gebiets mit mittlerem Risiko variiert die Topographie. Höher gelegene Standorte weisen eine geringere Überflutungstiefe auf. Ähnlich verhält es sich mit 12 Datenpunkten in Gebieten mit niedrigem Risiko. Auch dort können nicht-null Überschwemmungstiefen auftreten. Dies ist auf die niedrigere Geländehöhe zurückzuführen. 

Die Ermittlung der Tiefe für 3823 Datenpunkte in sehr niedrigen Risikogebieten wurde nicht durchgeführt. Diese Datenpunkte umfassen 1518 Orte, verteilt über 72 Landkreise. Eine individuelle Bewertung ist sehr aufwendig, da alle Kartendaten manuell geladen werden müssen und kein vollständiger Datensatz für Bayern mit Pegelnullpunkten und Hochwasserständen vorhanden ist. Diese Informationen müssen zudem auch manuell gesucht werden. Aufgrund dieser Einschränkungen wurden für alle Datenpunkte in sehr niedrigen Risikogebieten eine Überflutungstiefe und Schadensfaktor von 0 angenommen.

Abbildung \ref{fig:schadenwert} zeigt eine rechtsschiefe Verteilung des Immobilienschadens, wobei die Mehrheit der Fälle im Bereich von etwa 8.000 bis 20.000 liegt. Der Median von 23.275,68 € liegt signifikant unter dem arithmetischen Durchschnitt von 35.558,03 €, was darauf hindeutet, dass es eine Ausreißer mit sehr hohen Schadensbeträgen gibt.

Die Analyse zeigt signifikante Veränderungen in mehreren Finanzkennzahlen. Der durchschnittliche Loan-to-Value stieg von 43,31\% auf 47,85\%, was auf ein erhöhtes Kreditrisiko hindeutet. Dies resultiert aus der Wertminderung der Sicherheiten bei unverändertem Kreditbetrag. Der mittlere Immobilienwert sank von 475.937,88€ auf 440.379,85€, was die negativen Auswirkungen von Hochwasserrisiken auf den Vermögenswert verdeutlicht. Diese Entwicklung war in allen untersuchten Fällen zu beobachten. Die Risk-Weighted Assets zeigten einen leichten Anstieg im Durchschnitt von 41.952,93€ auf 43.140,48€. Obwohl dieser Anstieg nur in 15,38\% der Fälle auftrat, könnte er zu erhöhten Kapitalanforderungen für die Bank führen. Die durchschnittliche RWA-Änderung betrug 3,46\%. Diese Trends reflektieren sowohl Veränderungen im Immobilienmarkt als auch mögliche Anpassungen in der Risikobewertung und Kapitalpolitik der Finanzinstitute.