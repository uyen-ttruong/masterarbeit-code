
\section{Diskussion}\label{kap:6}
\subsection{Datensatz}

Abbildung \ref{fig:hypothekenportfolio} demonstriert eine erfolgreiche Verteilung der Datenpunkte basierend auf der Bevölkerungsdichte. In dunkelgrün markierten Gebieten, die eine hohe Bevölkerungsdichte repräsentieren, ist eine Konzentration roter Punkte zu beobachten. Diese roten Punkte symbolisieren Immobilien.
Die Darstellung der Hochwasserrisikogebiete in Abbildung \ref{fig:bayernflut} korrespondiert mit den realen Gegebenheiten. Städte wie München, Ingolstadt, Augsburg, Regensburg, Kempten und Passau befinden sich in der Nähe großer Flüsse. Historische Daten belegen, dass in diesen Städten, mit Ausnahme von München, bereits schwere Überschwemmungen aufgetreten sind.

Tabelle \ref{tab:hypothekenuberblick} bietet einen umfassenden Überblick über die wesentlichen Finanzkennzahlen und statistischen Merkmale des analysierten Hypothekenportfolios. Die Darlehenbeträge, die im Mittel bei 168.148,10 € liegen, entsprechen nahe der durchschnittlichen Darlehensgröße für Wohnimmobilien der Münchener Hypothekenbank von etwa 163.700 €. Die Quadratmeterpreise variieren erheblich, mit einem Durchschnitt von 2.458,57 € und einer Standardabweichung von 1.320,59 €, was die Diversität des Immobilienmarktes widerspiegelt. Die Wohnflächen im Portfolio reichen von 80,00 m² bis 168,23 m², mit einem Durchschnitt von 114,40 m² und einer Standardabweichung von 29,14 m². Diese Verteilung zeigt, dass das Portfolio verschiedene Wohnungsgrößen umfasst, wobei der Schwerpunkt auf mittelgroßen bis größeren Wohneinheiten liegt.

Der durchschnittliche aktuelle Immobilienwert liegt bei 281.726,31 €. In Verbindung mit dem durchschnittlichen Beleihungsauslauf von 59,78\% lässt sich auf eine konservative Kreditvergabepraxis schließen. Dieses \ac{LTV}-Verhältnis deutet nicht nur auf eine vorsichtige Risikoeinschätzung seitens der Bank hin, sondern auch auf eine solide Eigenkapitalbasis der Kreditnehmer. Die Spannbreite der Darlehensbeträge, von 26.368,90 € bis 662.763,36 €, zeigt zudem, dass das Portfolio sowohl kleinere als auch größere Finanzierungen umfasst, was zur Risikodiversifizierung beiträgt.

\subsection{Physisches Risiko}
Aufgrund der Größe der \ac{DGM}-Daten (240 GB) erfolgte die Berechnung für 30 Punkte in hohem-, mittlerem und niedrigem Risikogebiet. Entsprechende Orts-, Gemeinde- und Landkreisdaten wurden geladen.
Im Hochrisikogebiet liegt ein einzelner Punkt in Haag an der Amper. Dort beträgt die maximale Überflutungstiefe 2,1 m, entsprechend einem Schadensfaktor von 0,16.
17 Datenpunkte befinden sich in Gebieten mittleren Risikos, verteilt auf verschiedene Orte. In der Kategorie mittleres Risiko gibt es Punkte mit einer Überschwemmungstiefe von 0. Innerhalb eines Gebiets mit mittlerem Risiko variiert die Topographie. Höher gelegene Standorte weisen eine geringere Überflutungstiefe auf. Ähnlich verhält es sich mit 12 Datenpunkten in Gebieten mit niedrigem Risiko. Auch dort können nicht-null Überschwemmungstiefen auftreten. Dies ist auf die niedrigere Geländehöhe zurückzuführen. 

Die Ermittlung der Tiefe für 3823 Datenpunkte in sehr niedrigen Risikogebieten wurde nicht durchgeführt. Diese Datenpunkte umfassen 1518 Orte, verteilt über 72 Landkreise. Es ist zeitaufwändig, eine individuelle Bewertung durchzuführen, da sämtliche Kartendaten manuell geladen werden müssen und es keinen vollständigen Datensatz für Bayern mit Pegelnullpunkten und Hochwasserständen gibt. Diese Informationen müssen zudem manuell gesucht werden. Aufgrund dieser Einschränkungen wurden für alle Datenpunkte in sehr niedrigen Risikogebieten eine Überflutungstiefe und Schadensfaktor von 0 angenommen.

Abbildung \ref{fig:schadenwert} zeigt eine Verteilung des Immobilienschadens mit einer Tendenz zur Linksschiefe. Der kleinste dargestellte Schadenwert liegt bei etwa 7.200 €. Der höchste liegt bei ungefähr 15.000 €, mit einer Häufigkeit von 6 Fällen. Zwischen 27.000 € und 42.000 € gibt es eine weitere Ansammlung von insgesamt 3 Fällen. Die Verteilung weist weitere vereinzelte Fälle bei höheren Werten auf, insbesondere um 60.000 € und über 100.000 €, jeweils mit einer Häufigkeit von etwa 1 Fall.  Der Median von 15.378,16 € liegt deutlich unter dem arithmetischen Mittelwert von 29.101,00 €, was auf das Vorhandensein von Ausreißern mit hohen Schadensbeträgen hinweist. Diese Verteilung spiegelt eindeutig eine stark differenzierte Struktur der Schadenshöhen wider. Sie entspricht der realen Komplexität des Immobilienmarktes und wird maßgeblich durch verschiedene Faktoren beeinflusst, insbesondere durch Immobilientypen, spezifische Schadensarten und deutliche regionale Unterschiede.


Die vorliegenden Kreisdiagramme \ref{fig:schadenereignis} und \ref{fig:schadenereignis_jahr} visualisieren das Verhältnis zwischen Gesamtschadenwert und Immobilienwert in verschiedenen Kontexten. Die Summe der Immobilienwerte der von Hochwasserereignissen betroffenen Objekte beläuft sich auf 4.701.174,77 € und der ermittelte Gesamtschadenwert ist 378.312,95 €, welcher 8\% des kumulierten Immobilienwertes repräsentiert. Von insgesamt 3.853 Immobilien mit einem kumulierten Wert von 1.085.491.467,75 € sind 13 Objekte von Hochwasserschäden betroffen, was den prozentualen Anteil des Schadenwertes am Gesamtportfoliowert auf  0,03\% reduziert. Das Diagramm \ref{fig:schadenereignis_jahr} zeigt einen relativen Schadenwert von 0,2\% bezogen auf den Wert der betroffenen Immobilien. Diese Reduktion resultiert aus der Integration der Jährlichen Überschreitungswahrscheinlichkeit \ac{AEP}, welche die Eintrittswahrscheinlichkeit von Schadensereignissen innerhalb eines Jahres berücksichtigt.

Die Tabelle \ref{tab:rwa_anderung} verdeutlicht, dass nach einem Schadensereignis der Beleihungsauslauf ansteigt, was in vier Fällen zu einer Erhöhung des Risikogewichts  führt. Die Zunahme der Neue-\ac{RWA}-Summe 699.556,26 € gegenüber der Aktuelle-\ac{RWA}-Summe 656.952,48 € zeigt, dass die Bank nach dem Schadenereignis stärker belastet wird und zusätzliche Kapitalreserven benötigt, um die erhöhten Risiken zu decken.


\subsection{Transitionsrisiko}

Im Netto-Null-Szenario steigt der Endenergiepreis von 0,058 €/kWh im Jahr 2023 deutlich an und erreicht bis 2050 etwa 0,34 €/kWh. Dies spiegelt die strengen Maßnahmen zur Emissionsreduktion wider. Im Ungeordnet-Szenario verläuft der Anstieg moderater, wobei der Preis im Jahr 2050 bei etwa 0,26 Euro/kWh liegt. Das Szenario Unter 2°C zeigt ebenfalls einen gleichmäßigen Anstieg auf etwa 0,16 €/kWh bis 2050. Im Gegensatz dazu bleibt der Preis im Aktuelle Richtlinien-Szenario nahezu konstant, leicht steigend von 0,058 €/kWh im Jahr 2023 auf 0,059 €/kWh im Jahr 2050.

In den vier dargestellten Szenarien zeigt sich, dass der Immobilienwert in unterschiedlichem Maße abnimmt, je nach Energieeffizienzklasse und künftigen Energiepreisen. 

Das Szenario Netto-Null weist die größten Wertverluste auf, insbesondere ab 2040. Immobilien der Energieklasse H verlieren fast 40\% ihres Werts. Dies steht im direkten Zusammenhang mit dem Anstieg des Endenergiepreises im Netto-Null-Szenario, der bis 2050 auf etwa 0,35 €/kWh ansteigt und ineffiziente Gebäude betrifft. Dies belegt die Auswirkungen der Klimaschutzmaßnahmen zur Reduzierung der Kohlenstoffemissionen, die erheblichen Druck auf die Immobilienwerte ausüben, insbesondere bei energieineffizienten Gebäuden.

Im Szenario Ungeordnet sinken die Immobilienwerte ab 2030 deutlich, wobei besonders die Energieklasse H den größten Wertverlust verzeichnet. Dies stimmt mit dem starken Anstieg des Endenergiepreises im Disorderly-Szenario überein. Der Endenergiepreis steigt ab 2040 stark an, was den hohen Wertverlust bei ineffizienten Immobilien erklärt. Die Wertverluste sind jedoch weniger ausgeprägt als im Netto-Null-Szenario, was die Auswirkungen einer instabilen Klimapolitik verdeutlicht.

Im Szenario Aktuelle Richtlinien bleiben die Immobilienwerte relativ stabil. Bis 2050 sind nur geringe Wertverluste zu beobachten, was durch die Konstanz des Endenergiepreises im Szenario Current Policies unterstützt wird. Die Grafik zeigt sechs verschiedene Szenarien (B, C, D, F, G, H), wobei alle einen ähnlichen Verlauf mit einem leichten Höhepunkt um 2030 und einem anschließenden moderaten Abwärtstrend aufweisen. Trotz der allgemeinen Abwärtstendenz ab 2035 bleiben die Wertverluste begrenzt, und zum Ende des Betrachtungszeitraums ist eine leichte Erholung erkennbar, obwohl die Werte im negativen Bereich verbleiben. Dies spiegelt wider, dass die aktuellen Politiken die Immobilienwerte wenig beeinflussen und verdeutlicht die relative finanzielle Stabilität ohne strenge Klimaschutzmaßnahmen.

Im Szenario Unter 2°C beginnen die Immobilienwerte ab 2030 zu fallen, wobei die Energieklasse H bis zu 14\% ihres Werts verliert. Der moderate Anstieg des Endenergiepreises spiegelt diese Verluste wider und verdeutlicht den Zusammenhang zwischen steigenden Energiepreisen und sinkenden Immobilienwerten. Immobilien mit niedrigerer Energieeffizienz (Klassen G und H) verlieren mehr an Wert, jedoch nicht so stark wie im Netto-Null-Szenario.


Die Abbildung \ref{fig:gesamt_rwa_plot} zeigt vier divergierende \acs{RWA}-Entwicklungen von 2020 bis 2050, wobei das Netto-Null-Szenario den steilsten Anstieg aufweist, gefolgt vom ungeordneten Szenario und dem Unter-2°C-Szenario, während die aktuellen Richtlinien zu einem nahezu konstanten \acs{RWA}-Wert führen. Ab 2025 beginnen die Szenarien deutlich zu divergieren, wobei das Netto-Null-Szenario 2050 den höchsten \acs{RWA}-Wert erreicht, deutlich über den anderen Szenarien. Diese Divergenz impliziert, dass Banken je nach Klimaszenario mit sehr unterschiedlichen Kapitalanforderungen konfrontiert sein könnten und ihr Risikomanagement sowie ihre Kapitalplanung langfristig an verschiedene Klimaszenarien anpassen müssen.