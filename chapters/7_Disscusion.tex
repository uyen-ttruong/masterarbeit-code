
\section{Diskussion}
\subsection{Datensatz}

Abbildung \ref{fig:hypothekenportfolio} demonstriert eine erfolgreiche Verteilung der Datenpunkte basierend auf der Bevölkerungsdichte. In dunkelgrün markierten Gebieten, die eine hohe Bevölkerungsdichte repräsentieren, ist eine Konzentration roter Punkte zu beobachten. Diese roten Punkte symbolisieren Immobilien.
Die Darstellung der Hochwasserrisikogebiete in Abbildung \ref{fig:bayernflut} korrespondiert mit den realen Gegebenheiten. Städte wie München, Ingolstadt, Augsburg, Regensburg, Kempten und Passau befinden sich in der Nähe großer Flüsse. Historische Daten belegen, dass in diesen Städten, mit Ausnahme von München, bereits schwere Überschwemmungen aufgetreten sind.

Tabelle \ref{tab:hypothekenuberblick} bietet einen umfassenden Überblick über die wesentlichen Finanzkennzahlen und statistischen Merkmale des analysierten Hypothekenportfolios. Die Darlehenbeträge, die im Mittel bei 168.148,10 € liegen, entsprechen nahe der durchschnittlichen Darlehensgröße für Wohnimmobilien der Münchener Hypothekenbank von etwa 163.700 €. Die Quadratmeterpreise variieren erheblich, mit einem Durchschnitt von 2.458,57 € und einer Standardabweichung von 1.320,59 €, was die Diversität des Immobilienmarktes widerspiegelt. Die Wohnflächen im Portfolio reichen von 80,00 m² bis 168,23 m², mit einem Durchschnitt von 114,40 m² und einer Standardabweichung von 29,14 m². Diese Verteilung zeigt, dass das Portfolio verschiedene Wohnungsgrößen umfasst, wobei der Schwerpunkt auf mittelgroßen bis größeren Wohneinheiten liegt.

Der durchschnittliche aktuelle Immobilienwert liegt bei 281.726,31 €. In Verbindung mit dem durchschnittlichen Beleihungsauslauf von 59,78\% lässt sich auf eine konservative Kreditvergabepraxis schließen. Dieses LTV-Verhältnis deutet nicht nur auf eine vorsichtige Risikoeinschätzung seitens der Bank hin, sondern auch auf eine solide Eigenkapitalbasis der Kreditnehmer. Die Spannbreite der Darlehensbeträge, von 26.368,90 € bis 662.763,36 €, zeigt zudem, dass das Portfolio sowohl kleinere als auch größere Finanzierungen umfasst, was zur Risikodiversifizierung beiträgt.

\subsection{Physische Riskio}
Aufgrund der Größe der \ac{DGM}-Daten (240 GB) erfolgte die Berechnung nur für 30 Punkte in Hochrisiko-, mittlerem und niedrigem Risikogebiet. Entsprechende Orts-, Gemeinde- und Landkreisdaten wurden geladen.
Im Hochrisikogebiet liegt ein einzelner Punkt in Haag an der Amper. Dort beträgt die maximale Überflutungstiefe 2,1 m, entsprechend einem Schadensfaktor von 0,16.
17 Datenpunkte befinden sich in Gebieten mittleren Risikos, verteilt auf verschiedene Orte.In der Kategorie mittleres Risiko gibt es Punkte mit einer Überschwemmungstiefe von 0. Dies ist durchaus plausibel. Innerhalb eines Gebiets mit mittlerem Risiko variiert die Topographie. Höher gelegene Standorte weisen eine geringere Überflutungstiefe auf. Ähnlich verhält es sich mit 12 Datenpunkten in Gebieten mit niedrigem Risiko. Auch dort können nicht-null Überschwemmungstiefen auftreten. Dies ist auf die niedrigere Geländehöhe zurückzuführen. 

Die Ermittlung der Tiefe für 3823 Datenpunkte in sehr niedrigen Risikogebieten wurde nicht durchgeführt. Diese Datenpunkte umfassen 1518 Orte, verteilt über 72 Landkreise. Eine individuelle Bewertung ist sehr aufwendig, da alle Kartendaten manuell geladen werden müssen und kein vollständiger Datensatz für Bayern mit Pegelnullpunkten und Hochwasserständen vorhanden ist. Diese Informationen müssen zudem auch manuell gesucht werden. Aufgrund dieser Einschränkungen wurden für alle Datenpunkte in sehr niedrigen Risikogebieten eine Überflutungstiefe und Schadensfaktor von 0 angenommen.

Abbildung \ref{fig:schadenwert} zeigt eine Verteilung des Immobilienschadens mit einer Tendenz zur Linksschiefe. Der kleinste dargestellte Schadenwert liegt bei etwa 7.200 €. Der höchste Gipfel liegt bei ungefähr 15.000 €, mit einer Häufigkeit von 6 Fällen. Zwischen 27.000 € und 42.000 € gibt es eine weitere Ansammlung von insgesamt 3 Fällen. Die Verteilung weist weitere vereinzelte Fälle bei höheren Werten auf, insbesondere um 60.000 € und über 100.000 €, jeweils mit einer Häufigkeit von etwa 1 Fall.  Der Median von 15.378,16 € liegt deutlich unter dem arithmetischen Mittelwert von 29.101,00 €, was auf das Vorhandensein von Ausreißern mit hohen Schadensbeträgen hinweist. Diese Verteilung spiegelt eindeutig eine stark differenzierte Struktur der Schadenshöhen wider. Sie entspricht der realen Komplexität des Immobilienmarktes und wird maßgeblich durch verschiedene Faktoren beeinflusst, insbesondere durch Immobilientypen, spezifische Schadensarten und signifikante regionale Unterschiede.


Die vorliegenden Kreisdiagramme \ref{fig:schadenereignis} und \ref{fig:schadenereignis_jahr} visualisieren das Verhältnis zwischen Gesamtschadenwert und Immobilienwert in verschiedenen Kontexten. Die Summe der Immobilienwerte der von Hochwasserereignissen betroffenen Objekte beläuft sich auf 4.701.174,77 €. In Relation zu diesem Aggregat steht der ermittelte Gesamtschadenwert von 378.312,95 €, welcher approximativ 8\% des kumulierten Immobilienwertes repräsentiert. Im Kontext des Gesamtportfolios ergibt sich jedoch eine divergierende Perspektive: Von insgesamt 3.853 Immobilien mit einem kumulierten Wert von 1.085.491.467,75 € sind lediglich 13 Objekte von Hochwasserschäden betroffen, was den prozentualen Anteil des Schadenwertes am Gesamtportfoliowert auf approximativ 0,03\% reduziert. Das Diagramm \ref{fig:schadenereignis_jahr} indiziert einen relativen Schadenwert von 0,2\% bezogen auf den Wert der betroffenen Immobilien. Diese signifikante Reduktion resultiert aus der Integration der Jährlichen Überschreitungswahrscheinlichkeit \ac{AEP}, welche die Eintrittswahrscheinlichkeit von Schadensereignissen innerhalb eines Jahres berücksichtigt.

Die Abbildung \ref{fig:schadenLtV} stellt die aktuellen und neuen Beleihungsausläufe für Immobilien gegenüber. In allen dargestellten Fällen, die sich auf jede von Überschwemmungen betroffene Immobilie beziehen, ist der neue Beleihungsauslauf größer. Wenn der Betrag des Darlehens gleich bleibt, führt ein Rückgang der Immobilienwerte zu einem höheren Beleihungsauslauf-Verhältnis. Ein erhöhter Beleihungsauslauf stellt ein höheres Risiko für Banken dar.

\subsection{Transitionsrisiko}

Im Netto-Null-Szenario steigt Endenergiepreis von 0,058 Euro/kWh im Jahr 2023 deutlich an und erreicht bis 2050 etwa 0,34 Euro/kWh. Dies spiegelt die strengen Maßnahmen zur Emissionsreduktion wider. Im Ungeordnet-Szenario verläuft der Anstieg moderater, wobei der Preis im Jahr 2050 bei etwa 0,26 Euro/kWh liegt. Das Szenario Unter 2°C zeigt ebenfalls einen gleichmäßigen Anstieg auf etwa 0,16 Euro/kWh bis 2050. Im Gegensatz dazu bleibt der Preis im Aktuelle Richtlinien-Szenario nahezu konstant, leicht steigend von 0,058 Euro/kWh im Jahr 2023 auf 0,059 Euro/kWh im Jahr 2050.

In den vier dargestellten Szenarien zeigt sich, dass der Immobilienwert in unterschiedlichem Maße abnimmt, je nach Energieeffizienzklasse und künftigen Energiepreisen. Im Szenario Ungeordnet sinken die Immobilienwerte ab 2030 deutlich, wobei besonders die Energieklasse H den größten Wertverlust verzeichnet. Dies korrespondiert mit dem starken Anstieg des Endenergiepreises im "Disorderly"-Szenario, wie in Abbildung \ref{fig
} zu sehen ist. Der Endenergiepreis steigt hier ab 2040 stark an, was den hohen Wertverlust bei ineffizienten Immobilien erklärt.

Im Szenario Aktuelle Richtlinien bleiben die Immobilienwerte relativ stabil. Bis 2050 sind nur geringe Wertverluste zu beobachten, was durch die Konstanz des Endenergiepreises im Szenario "Current Policies" unterstützt wird. Der Preis bleibt hier fast unverändert, was die Stabilität der Immobilienwerte erklärt.

Im Szenario Unter 2°C beginnen die Immobilienwerte ab 2030 zu fallen, wobei die Energieklasse H bis zu 14\% ihres Werts verliert. Der moderate Anstieg des Endenergiepreises im "2 Degrees"-Szenario spiegelt diese Verluste wider und verdeutlicht den Zusammenhang zwischen steigenden Energiepreisen und sinkenden Immobilienwerten.

Das Szenario Netto-Null weist die größten Wertverluste auf, insbesondere ab 2040. Immobilien der Energieklasse H verlieren fast 40\% ihres Werts. Dies steht im direkten Zusammenhang mit dem dramatischen Anstieg des Endenergiepreises im "Netto-Null"-Szenario, der bis 2050 auf etwa 0,35 Euro/kWh ansteigt und ineffiziente Gebäude besonders hart trifft.