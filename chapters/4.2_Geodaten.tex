%cSpell:disable
\subsection{Geodaten zu Hypotheken und Hochwasser}
\subsubsection{Hypothekengeodaten}\label{sec:hypogeo}

Für die Kompatibilität von Hypotheken- und Hochwasserdaten sind geografische Koordinaten erforderlich. Dieser Abschnitt befasst sich mit der Generierung präziser Koordinaten für die Datenpunkte.

Eine zufällige Verteilung in Bayern würde die Struktur eines Kreditportfolios nicht korrekt abbilden, da eine ungleichmäßige Verteilung von Immobilien sowohl in Deutschland als auch in Bayern zu beobachten ist. \textcite{zurek2022real} analysierte die Beziehung zwischen Bevölkerungsdichte und Kreditvergabe in Deutschland. Die Studie zeigt, dass Regionen mit stärkerem Wirtschaftswachstum höhere Immobilienpreise aufweisen. Dies führt zu einer erhöhten Kreditnachfrage. Auf Basis dieser empirischen Erkenntnisse wird die Bevölkerungsdichte als Grundlage für die Zuweisung spezifischer Koordinaten zu jedem Datenpunkt herangezogen.

Die verwendete Datenquelle stammt von \textcite{suche_postleitzahl}. Sie kombiniert OpenStreetMap-Daten mit Einwohnerzahlen von \textcite{destatis}. Dies ermöglicht eine präzise Segmentierung in Postleitzahlenzonen. Tabelle \ref{tab:geodaten} zeigt die Daten dieser geographischen Strukturierung.
Übersicht der Geodaten für Postleitzahlgebiete in Bayern
\begin{table}[htbp]
    \centering
    \small
    \caption{Übersicht der Geodaten für Postleitzahlgebiete in Bayern}
    \label{tab:geodaten}
    \begin{tabularx}{1.0\textwidth}{>{\raggedright\arraybackslash}X >{\raggedright\arraybackslash}X}
        \toprule
        \textbf{Objekt} & \textbf{Erklärung} \\
        \midrule
        plz & Postleitzahl \\
        \addlinespace
        einwohner & Die Einwohnerzahl eines bestimmten Ortes \\
        \addlinespace
        qkm & Die Fläche des Gebiets in Quadratkilometern \\
        \addlinespace
        geometry & Die Koordinaten des Gebiets \\
        \addlinespace
        ort & Der Name des Ortes, in dem sich das Gebiet befindet \\
        \addlinespace
        landkreis & Die Zugehörigkeit zu einem Landkreis \\
        \bottomrule
    \end{tabularx}
\end{table}
\FloatBarrier


Tabelle \ref{tab:geodaten} wurde aus Shapefile-Daten der Postleitzahlenregionen generiert und umfasst die Bevölkerungsverteilung. Vier Spalten sind von besonderer Relevanz: Ort, Landkreis, Geometrie und Einwohner. Die Geometriespalte enthält die geografischen Koordinaten der Gemeinden, dargestellt als Polygon oder Multipolygon. Ein Multipolygon setzt sich aus mehreren Einzelpolygonen verschiedener Formen zusammen. Zur räumlichen Referenzierung dient das Koordinatensystem EPSG:3035. Abbildung \ref{fig:bevoelkerungsdichte} visualisiert die aus diesen Daten abgeleitete Bevölkerungsdichteverteilung Bayerns nach Postleitzahlenbereichen.

\begin{figure}[htbp]
    \centering
    \includegraphics[width=0.95\textwidth]{figures/Bayern_pop_plz.png}
    \caption{Verteilung der Bevölkerungsdichte Bayerns nach Postleitzahlenbereichen. Quelle: Eigene Darstellung}
    \label{fig:bevoelkerungsdichte}
\end{figure}
\FloatBarrier



\subsubsection{Hochwassergeodaten}\label{sec:hochgeo}

Im Anschluss an die in Abschnitt \ref{sec:hypogeo} dargelegte Erfassung der geografischen Koordinaten der Hypothekendarlehen ergibt sich die Notwendigkeit einer weiterführenden Analyse. Diese zielt darauf ab, die räumliche Relation der betreffenden Immobilien zu den definierten Hochwasserrisikogebieten zu determinieren. Zur Durchführung dieser Analyse ist eine detaillierte Hochwasserrisikokarte für Bayern erforderlich.

Im Rahmen eines EU-weiten Stresstests stellt die \ac{EZB} den Banken zur Simulation eines schweren Überschwemmungsszenarios eine Hochwasserrisikokarte (Abbildung \ref{fig:euflut}) zur Verfügung.

\begin{figure}[htbp]
    \centering
    \includegraphics[width=0.6\textwidth]{figures/euflood.png} 
    \caption{EU-Hochwasserrisikokarte.Quelle: EZB 2022 Klimarisiko-Stresstest}
    \label{fig:euflut}
\end{figure}
\FloatBarrier

Die \ac{EZB}-Karte klassifiziert Regionen nach Hochwasserrisiken. Sie basiert auf Daten der Europäischen Kommission und Four Twenty Seven \parencite{ECB2022ClimateStressTest}. Allerdings sind die Basisdaten nicht öffentlich zugänglich. Zudem erlaubt die visuelle Darstellung keine präzise Identifikation spezifischer Gebiete. Folglich ergibt sich für Bayern die Notwendigkeit einer detaillierteren Analyse.

Die Kartierungen des Bayerischen Landesamts für Umwelt bieten eine fundierte Grundlage für die regionale Risikoanalyse. Diese für die Einzugsgebiete von Donau, Rhein und Elbe entwickelten Karten ermöglichen eine präzise Einschätzung der Hochwasserrisiken in Bayern \parencite{LfU_Bayern}. Sie visualisieren detailliert die Hochwassergefährdung, potenziell betroffene Landnutzungen und historische Hochwasserereignisse.
Diese Daten liegen im ETRS89-Koordinatensystem vor, während die Hypothekengeodaten das EPSG:3035-System nutzen. ETRS89 ist ein geodätisches Referenzsystem für Europa. Es misst Positionen in geografischen Koordinaten: Breitengrad und Längengrad. Diese werden in Grad, Minuten und Sekunden angegeben. ETRS89 bietet eine hohe Genauigkeit für kontinentale Messungen.
EPSG:3035 hingegen ist eine kartografische Projektion. Sie wandelt die Erdkrümmung in eine flache Ebene um. Positionen werden hier in Metern gemessen. X-Koordinaten repräsentieren den Abstand vom Projektionszentrum nach Osten. Y-Koordinaten messen den Abstand nach Norden. EPSG:3035 ist speziell für statistische Analysen in Europa konzipiert.
Zur Herstellung der Datenkohärenz erfolgt eine Koordinatentransformation in das EPSG:3035-System. 

\subsubsection{Überflutungstiefen}\label{sec:tief}
Die Schäden durch Überschwemmungen hängen von der Wassertiefe ab. Tieferes Wasser verursacht meist größere und teurere Schäden an Häusern. Für die Analyse der Überflutungstiefe in Bayern benötigt man Überflutungstiefen-Geodaten. \ac{DGM} Modell beschreibt die Höhe des Bodens, wobei eine hohe Auflösung genaue Ergebnisse liefert \parencite{vermessungsverwaltung2019gelandemodell}. Hochwasserstände aus Messungen sind wichtig für präzise Berechnungen. Dafür wurden Daten vom \textcite{bayern2016hochwassernachrichtendienst} für aktuelle Pegelstände betrachtet. Historische Hochwasserdaten helfen bei Einschätzungen künftiger Ereignisse und sind in Berichten und Karten enthalten. Diese wurden vom \textcite{LfU_Bayern} bezogen, wie in Abschnitt \ref{sec:hochgeo} beschrieben.
Die \ac{DGM}-Daten von \textcite{vermessungsverwaltung2019gelandemodell} für ganz Bayern umfassen ein sehr großes Datenvolumen von ca. 240 GB. Daher wurden nur Daten für die Gebiete mit Hypotheken-Datenpunkten aus Abschnitt \ref{sec:hypogeo} heruntergeladen.
Die Geländehöhe (m) für bestimmte Koordinaten wird aus dem Höhenraster, das aus der \ac{DGM}-Datei gelesen wurde, extrahiert.

Auf Grundlage der gesammelten Daten können die folgenden Berechnungen durchgeführt werden:
\begin{equation}
    \text{Absoluter Wasserstand (m)} = \text{Pegelnullpunkt (m)} + \left(\frac{\text{Pegelstand (cm)}}{100}\right)
\end{equation}

\begin{equation}
    \text{Hochwassertiefe (m)} = \max \left( \text{Wasserstand (m)} - \text{Geländehöhe (m)}, 0 \right)
\end{equation}
 Abbildung \ref{fig:ingolstadt} visualisiert das digitale Geländemodell für Ingolstadt, welches die topographischen Merkmale der Stadt und ihrer Umgebung detailliert darstellt und somit eine wichtige Grundlage für die Hochwasseranalyse bildet.
\begin{figure}[!ht]
    \centering
    \includegraphics[width=0.8\textwidth]{figures/dgm_3d_wireframe_ingolstadt.png}
    \caption{Digital Geländemodell von Ingolstadt. Quelle: Eigene Darstellung}
    \label{fig:ingolstadt}
\end{figure}
\FloatBarrier