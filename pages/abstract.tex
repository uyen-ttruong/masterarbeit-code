

\section*{Abstract}
Diese Arbeit entwickelt eine Methodik zur Integration von Klimarisiken und geospatialer Analyse in die Risiko Bewertung von Immobilienportfolios. Der Schwerpunkt liegt auf Immobilien in Bayern, wobei sowohl physische als auch Transitionsrisiken analysiert wurden. Ein repräsentatives Hypothekenportfolio wurde erstellt, basierend auf Daten der Münchener Hypothekenbank und öffentlichen Quellen von Bayerrisches Landesamt für Umwelt und Bayerischen Vermessungsverwaltung. Physische Risiken, insbesondere Hochwasser, wurden mittels digitaler Geländemodelle und Schadensfunktionen quantifiziert. Transitionsrisiken wurden anhand von NGFS-Klimaszenarien und Energieeffizienzklassen bewertet. Die Ergebnisse zeigen, dass Hochwasserrisiken selektiv, aber potenziell signifikant sind. Energiepreisänderungen könnten erhebliche Auswirkungen auf Immobilienwerte haben, besonders bei ineffizienten Gebäuden. Die Arbeit bietet eine Grundlage für präzisere Risikomodelle im Immobiliensektor und unterstreicht die Bedeutung der Klimarisikointegration in Finanzentscheidungen.



