

\section*{Abstract}
Diese Arbeit entwickelt eine Methodik zur Integration von Klimarisiken und geospatialer Analyse in die Risiko Bewertung von Immobilienportfolios. Der Schwerpunkt liegt auf Immobilien in Bayern, wobei sowohl physische als auch Transitionsrisiken analysiert wurden. Ein repräsentatives Hypothekenportfolio wurde erstellt, basierend auf Daten der Münchener Hypothekenbank und öffentlichen Quellen von dem bayerrischen Landesamt für Umwelt und der bayerischen Vermessungsverwaltung. Physische Risiken, insbesondere Hochwasser, wurden mittels digitaler Geländemodelle und Schadensfunktionen quantifiziert. Transitionsrisiken wurden anhand von \ac{NGFS}-Klimaszenarien und Energieeffizienzklassen bewertet. Die Ergebnisse zeigen, dass Hochwasserrisiken nicht gleichmäßig über alle Immobilien verteilt sind. Sie konzentrieren sich auf bestimmte geografische Gebiete, wo ihre Auswirkungen erheblich sein können. Energiepreisänderungen könnten erhebliche Auswirkungen auf Immobilienwerte haben, besonders bei ineffizienten Gebäuden. Die Arbeit bietet eine Grundlage für Risikomodelle im Immobiliensektor und unterstreicht die Bedeutung der Klimarisikointegration in Finanzentscheidungen.



